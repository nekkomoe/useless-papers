%!TEX program = xelatex

\documentclass[UTF8]{article}
\author {sjzez czy}
\title {简要题解}
\date{2019.1.23}
\usepackage[UTF8]{ctex}
\usepackage{listings}
\usepackage{fontspec}
\usepackage{ctex}
\usepackage{amsmath}
\usepackage{amssymb}
\usepackage{geometry}
\usepackage{setspace}
\usepackage{abstract}
\usepackage{graphicx}
\setmonofont{Consolas}
\usepackage{verbatim}
\usepackage[colorlinks,linkcolor=black,citecolor=black]{hyperref}
\renewcommand{\baselinestretch}{1.5}
\geometry{a4paper,left=2.7cm,right=2.7cm,top=2.7cm,bottom=2.7cm}



\begin{document}

\maketitle

\tableofcontents

\newpage

%%% 模板
\iffalse 
\section{\href{https://lydsy.com}{题目名称}}
\subsection{题目大意}
\subsection{算法讨论}
\subsection{时间复杂度}
\subsection{空间复杂度}
\fi

\section{\href{https://www.51nod.com/Challenge/Problem.html?problemId=1172}{【51nod 1172】Partial Sums V2}}

\subsection{题目大意}

给定一个序列(长度不超过 $50000$),求做 $k(k \le 10^9)$ 次前缀和后的序列结果,序列的每个元素对 $10^9+7$ 取模

\subsection{算法讨论}

对于形式幂级数 $A(x)$,以及实数 $p$ 来说,有下式成立:

$$
F(x)=A(x)^p \Rightarrow A(x)F'(x)=pF(x)A'(x)
$$

对于一个序列 $\{a_n\}$ 来说,求一次前缀和相当于把 $A(x)$ 变为 $\frac{A(x)}{1-x}$

若做 $k$ 次前缀和,相当于乘上 $\frac{1}{(1-x)^k}$,由于乘法具有结合律,只需要考虑后式即可

令 $F(x)=\sum_{n=0}^{\infty}f_nx^n=\frac{1}{(1-x)^k}$,则有:

$$
\begin{aligned}
&F(x)=\frac{1}{(1-x)^k}=(1-x)^{-k} \\
\Rightarrow &(1-x)F'(x)=-kF(x)(-1)=kF(x) \\
\Rightarrow &\sum_{n=0}^{\infty}f_{n+1}(n+1)x^n-\sum_{n=1}^{\infty}f_nnx^n=\sum_{n=0}^{\infty}kf_nx^n \\
\end{aligned}
$$

即:

$$
\begin{cases}
f_0=F(0)=1 \\
f_{0+1}(0+1)=kf_0 \Rightarrow f_1=k \\
f_{n+1}(n+1)-f_{n}n=kf_n \quad (n \ge 1)
\end{cases}
$$

最后一个即:

$$
\begin{aligned}
&f_{n+1}=\frac{(k+n)f_n}{n+1} \quad & (n \ge 1) \\
\Rightarrow &f_{n}=\frac{(k+n-1)f_{n-1}}{n} \quad & (n \ge 2) \\
\end{aligned}
$$

因此可以 $O(n)$ 计算出 $\{f_n\}$ 后,再计算 $A(x) \times F(x)$ 即可

于是问题转化为了,给定两个序列,求其卷积,其中序列长度不超过 $50000$,运算在模 $10^9+7$ 意义下

如果使用分治乘法,时间复杂度为 $O(n^{1.59})$,代入数据可得跑的过去

\subsection{时间复杂度}

$$
O(n \log n)
$$

\subsection{空间复杂度}

$$
O(n)
$$

\section{\href{https://code.mi.com/problem/list/view?id=118}{【mioj 118】Grizzly and GCD}}

\subsection{题目大意}

给定 $n,a,b$,保证 $\gcd(a,b)=1$,且 $1 < n,a,b < 10^5$,求下式在模 $10^9+7$ 意义下的值:

$$
\sum_{m=0}^{n}[2 \not| {n \choose m}] \sum_{i=1}^{n}\sum_{j=1}^{i-1} \gcd(a^i-b^i,a^j-b^j)
$$

\subsection{算法讨论}

发现这就是个二合一,首先 $[2 \not| {n \choose m}]=[n \& m = m]$,之后考虑计算后面那个式子

需要用到一个结论,即:

$$
\gcd(a,b)=1 \Rightarrow \gcd(a^i-b^i,a^j-b^j)=a^{\gcd(i,j)}-b^{\gcd(i,j)}
$$

于是就相当于求:

$$
\sum_{i=1}^{n}\sum_{j=1}^{i-1}\left(a^{\gcd(i,j)}-b^{\gcd(i,j)}\right)
$$

也就是相当于求:

$$
\begin{aligned}
 &\sum_{i=1}^{n}\sum_{j=1}^{i-1}a^{\gcd(i,j)} \\
=&\sum_{d=1}^{n}a^d\sum_{i=1}^{\lfloor \frac{n}{d} \rfloor}\sum_{j=1}^{i-1}[\gcd(i,j)=1] \\
=&\sum_{d=1}^{n}a^d\left(-1+\sum_{i=1}^{\lfloor \frac{n}{d} \rfloor}\phi(i)\right)
\end{aligned}
$$

直接暴力就好了

\subsection{时间复杂度}

$$
O(n)
$$

\subsection{空间复杂度}

$$
O(n)
$$

\section{\href{https://www.zhixincode.com/contest/16/problem/G?problem_id=243}{【CCPC-Wannafly Winter Camp Day4 (Div1, onsite)】置置置换}}

\subsection{题目大意}

给定 $n$,求有多少个 $1 \sim n$ 的全排列,满足 $\forall 2 \le i \le n$,若 $2 \mid i$,则 $a_{i-1} > a_i$,否则 $a_{i-1} < a_i$

其中 $1 \le n \le 1000$,答案对 $10^9+7$ 取模

\subsection{算法讨论}

设 $f_n$ 表示 $1 \sim n$ 的满足条件的全排列的个数,则:

$$
\begin{cases}
f_0=1 \\
f_1=1 \\
f_n=\sum_{i=0}^{n-1} {n - 1 \choose i} f_i f_{n-1-i} [2 | i] \quad (n \ge 1)
\end{cases}
$$

意义就是考虑第 $n$ 个数是必须放到偶数位置上,也就是说 $n$ 左侧必须要有偶数个数字

\subsection{时间复杂度}

$$
O(n^2)
$$

\subsection{空间复杂度}

$$
O(n^2)
$$

\section{\href{https://www.zhixincode.com/contest/16/problem/I?problem_id=245}{【CCPC-Wannafly Winter Camp Day4 (Div1, onsite)】咆咆咆哮}}

\subsection{题目大意}

$wls$ 手上有 $n$ 张牌,每张牌他都可以选择召唤一个攻击力为$a_i$ 的生物,或者使得场上所有生物的攻击力加 $b_i$ 

请问如何抉择,使得场攻(场上生物攻击力的总和)最高

$wls$ 可以任意选择出这 $n$ 张牌的顺序

其中 $1 \le n \le 10^5,0 \le a_i,b_i \le 10^6$

\subsection{算法讨论}

首先最优决策一定是先召唤若干个生物,然后一直给它们加 $buff$

设 $f(x)$ 表示召唤出 $x$ 个生物时的最大攻击力,即有 $n-x$ 个 $buff$

由于某些原因,$f(x)$ 是一个关于 $x$ 的单峰函数,也就是可以三分

考虑 $f(x)$ 怎么求,假设已经决定了一些卡片是召唤,一些是加 $buff$,考虑一张召唤的卡片 $i$ 和 一张加 $buff$ 的卡片 $j$ 进行交换后答案更有的条件:

$$
a_j + b_i \times x > a_i + b_j \times x
$$

那么按照 $a_i-a_i \times x$ 降序排序后,前 $x$ 个卡片用于召唤,后 $n-x$ 个卡片用于加 $buff$ 就好了

\subsection{时间复杂度}

$$
O(n \log^2 n)
$$

\subsection{空间复杂度}

$$
O(n)
$$

\end{document}

