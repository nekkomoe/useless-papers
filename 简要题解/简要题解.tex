%!TEX program = xelatex

\documentclass[UTF8]{article}
\author {sjzez czy}
\title {简要题解}
\date{2019.1.23}
\usepackage[UTF8]{ctex}

\usepackage{listings}
\usepackage{fontspec}
\usepackage{amsmath}
\usepackage{amssymb}
\usepackage{geometry}
\usepackage{setspace}
\usepackage{abstract}
\usepackage{graphicx}
\usepackage{verbatim}
\usepackage[colorlinks,linkcolor=black,citecolor=black]{hyperref}
\renewcommand{\baselinestretch}{1.5}
\geometry{a4paper,left=2.7cm,right=2.7cm,top=2.7cm,bottom=2.7cm}
\setmonofont{Consolas}

\begin{document}

\maketitle

\tableofcontents

\newpage

%%% 模板
\iffalse 
\section{\href{https://lydsy.com}{题目名称}}
\subsection{题目大意}
\subsection{算法讨论}
\subsection{时间复杂度}
\subsection{空间复杂度}
\fi

\section{\href{https://www.51nod.com/Challenge/Problem.html?problemId=1172}{【51nod 1172】Partial Sums V2}}

\subsection{题目大意}

给定一个序列(长度不超过 $50000$),求做 $k(k \le 10^9)$ 次前缀和后的序列结果,序列的每个元素对 $10^9+7$ 取模

\subsection{算法讨论}

对于形式幂级数 $A(x)$,以及实数 $p$ 来说,有下式成立:

$$
F(x)=A(x)^p \Rightarrow A(x)F'(x)=pF(x)A'(x)
$$

对于一个序列 $\{a_n\}$ 来说,求一次前缀和相当于把 $A(x)$ 变为 $\frac{A(x)}{1-x}$

若做 $k$ 次前缀和,相当于乘上 $\frac{1}{(1-x)^k}$,由于乘法具有结合律,只需要考虑后式即可

令 $F(x)=\sum_{n=0}^{\infty}f_nx^n=\frac{1}{(1-x)^k}$,则有:

$$
\begin{aligned}
&F(x)=\frac{1}{(1-x)^k}=(1-x)^{-k} \\
\Rightarrow &(1-x)F'(x)=-kF(x)(-1)=kF(x) \\
\Rightarrow &\sum_{n=0}^{\infty}f_{n+1}(n+1)x^n-\sum_{n=1}^{\infty}f_nnx^n=\sum_{n=0}^{\infty}kf_nx^n \\
\end{aligned}
$$

即:

$$
\begin{cases}
f_0=F(0)=1 \\
f_{0+1}(0+1)=kf_0 \Rightarrow f_1=k \\
f_{n+1}(n+1)-f_{n}n=kf_n \quad (n \ge 1)
\end{cases}
$$

最后一个即:

$$
\begin{aligned}
&f_{n+1}=\frac{(k+n)f_n}{n+1} \quad & (n \ge 1) \\
\Rightarrow &f_{n}=\frac{(k+n-1)f_{n-1}}{n} \quad & (n \ge 2) \\
\end{aligned}
$$

因此可以 $O(n)$ 计算出 $\{f_n\}$ 后,再计算 $A(x) \times F(x)$ 即可

于是问题转化为了,给定两个序列,求其卷积,其中序列长度不超过 $50000$,运算在模 $10^9+7$ 意义下

如果使用分治乘法,时间复杂度为 $O(n^{1.59})$,代入数据可得跑的过去

\subsection{时间复杂度}

$$
O(n \log n)
$$

\subsection{空间复杂度}

$$
O(n)
$$

\section{\href{https://code.mi.com/problem/list/view?id=118}{【mioj 118】Grizzly and GCD}}

\subsection{题目大意}

给定 $n,a,b$,保证 $\gcd(a,b)=1$,且 $1 < n,a,b < 10^5$,求下式在模 $10^9+7$ 意义下的值:

$$
\sum_{m=0}^{n}[2 \not| {n \choose m}] \sum_{i=1}^{n}\sum_{j=1}^{i-1} \gcd(a^i-b^i,a^j-b^j)
$$

\subsection{算法讨论}

发现这就是个二合一,首先 $[2 \not| {n \choose m}]=[n \& m = m]$,之后考虑计算后面那个式子

需要用到一个结论,即:

$$
\gcd(a,b)=1 \Rightarrow \gcd(a^i-b^i,a^j-b^j)=a^{\gcd(i,j)}-b^{\gcd(i,j)}
$$

于是就相当于求:

$$
\sum_{i=1}^{n}\sum_{j=1}^{i-1}\left(a^{\gcd(i,j)}-b^{\gcd(i,j)}\right)
$$

也就是相当于求:

$$
\begin{aligned}
 &\sum_{i=1}^{n}\sum_{j=1}^{i-1}a^{\gcd(i,j)} \\
=&\sum_{d=1}^{n}a^d\sum_{i=1}^{\lfloor \frac{n}{d} \rfloor}\sum_{j=1}^{i-1}[\gcd(i,j)=1] \\
=&\sum_{d=1}^{n}a^d\left(-1+\sum_{i=1}^{\lfloor \frac{n}{d} \rfloor}\phi(i)\right)
\end{aligned}
$$

直接暴力就好了

\subsection{时间复杂度}

$$
O(n)
$$

\subsection{空间复杂度}

$$
O(n)
$$

\section{\href{https://www.zhixincode.com/contest/16/problem/G?problem_id=243}{【CCPC-Wannafly Winter Camp Day4 (Div1, onsite)】置置置换}}

\subsection{题目大意}

给定 $n$,求有多少个 $1 \sim n$ 的全排列,满足 $\forall 2 \le i \le n$,若 $2 \mid i$,则 $a_{i-1} > a_i$,否则 $a_{i-1} < a_i$

其中 $1 \le n \le 1000$,答案对 $10^9+7$ 取模

\subsection{算法讨论}

设 $f_n$ 表示 $1 \sim n$ 的满足条件的全排列的个数,则:

$$
\begin{cases}
f_0=1 \\
f_1=1 \\
f_n=\sum_{i=0}^{n-1} {n - 1 \choose i} f_i f_{n-1-i} [2 | i] \quad (n \ge 1)
\end{cases}
$$

意义就是考虑第 $n$ 个数是必须放到偶数位置上,也就是说 $n$ 左侧必须要有偶数个数字

\subsection{时间复杂度}

$$
O(n^2)
$$

\subsection{空间复杂度}

$$
O(n^2)
$$

\section{\href{https://www.zhixincode.com/contest/16/problem/I?problem_id=245}{【CCPC-Wannafly Winter Camp Day4 (Div1, onsite)】咆咆咆哮}}

\subsection{题目大意}

$wls$ 手上有 $n$ 张牌,每张牌他都可以选择召唤一个攻击力为$a_i$ 的生物,或者使得场上所有生物的攻击力加 $b_i$ 

请问如何抉择,使得场攻(场上生物攻击力的总和)最高

$wls$ 可以任意选择出这 $n$ 张牌的顺序

其中 $1 \le n \le 10^5,0 \le a_i,b_i \le 10^6$

\subsection{算法讨论}

首先最优决策一定是先召唤若干个生物,然后一直给它们加 $buff$

设 $f(x)$ 表示召唤出 $x$ 个生物时的最大攻击力,即有 $n-x$ 个 $buff$

由于某些原因,$f(x)$ 是一个关于 $x$ 的单峰函数,也就是可以三分

考虑 $f(x)$ 怎么求,假设已经决定了一些卡片是召唤,一些是加 $buff$,考虑一张召唤的卡片 $i$ 和 一张加 $buff$ 的卡片 $j$ 进行交换后答案更有的条件:

$$
a_j + b_i \times x > a_i + b_j \times x
$$

那么按照 $a_i-a_i \times x$ 降序排序后,前 $x$ 个卡片用于召唤,后 $n-x$ 个卡片用于加 $buff$ 就好了

\subsection{时间复杂度}

$$
O(n \log^2 n)
$$

\subsection{空间复杂度}

$$
O(n)
$$

\section{\href{https://www.51nod.com/Challenge/Problem.html?problemId=1627}{【51nod 1627】瞬间移动}}

\subsection{题目大意}

有一个无限大的矩形,初始时你在左上角(即第一行第一列)

每次你都可以选择一个右下方格子,并瞬移过去

求到第 $n(2 \le n \le 10^5)$ 行第 $m(2 \le m \le 10^5)$ 列的格子有几种方案,答案对 $1000000007$ 取模

\subsection{算法讨论}

枚举步数 $i$,之后相当于把 $n-1$ 分成 $i$ 份,$m-1$ 分成 $i$ 份

因此答案就是:

$$
\sum_{i=1}^{\min(n-1,m-1)} {n-2 \choose i-1} {m-2 \choose i-1}
$$

\subsection{时间复杂度}

$$
O(n)
$$

\subsection{空间复杂度}

$$
O(n)
$$

\section{\href{https://www.51nod.com/Challenge/Problem.html?problemId=1149}{【51nod 1149】Pi的递推式}}

\subsection{题目大意}

给定 $n(n \le 10^6)$,求 $f(n) \bmod (10^9+7)$,其中:

$$
\begin{aligned}
f(n)=\begin{cases}
1 &\quad 0 \le n < 4 \\
f(n-1) + f(n-\pi) &\quad n \ge 4
\end{cases}
\end{aligned}
$$

\subsection{算法讨论}

画出转移图,考虑 $f_i$ 对答案的贡献,枚举使用多少次 $\pi$,然后组合数计算答案

\subsection{时间复杂度}

$$
O(n)
$$

\subsection{空间复杂度}

$$
O(n)
$$

\section{\href{https://www.51nod.com/Challenge/Problem.html?problemId=1488}{【51nod 1488】帕斯卡小三角}}

\subsection{题目大意}

已知

$$
\begin{cases}
f_{1,j}=a_j &\qquad 1 \le j \le n \\
f_{i,j}=\min(f_{i-1,j},f_{i-1,j-1})+a_j &\qquad 2 \le i \le j \le n
\end{cases}
$$

其中 $a(a_i \le 10^4)$ 是一个长度为 $n(n \le 10^5)$ 的数组

有 $m(m \le 10^5)$ 次询问,输入 $x,y$,求 $f_{x,y}$

\subsection{算法讨论}

手玩后发现转移单调,即一定是从第一行的某个点往下走一段距离后一直往右下方走到目标点

写出动规方程,发现是斜率优化形式

由于凸包是静态的,可以直接线段树维护区间凸包,在每个凸包上三分就行

\subsection{时间复杂度}

$$
O(n \log^2 n)
$$

\subsection{空间复杂度}

$$
O(n \log n)
$$

\section{\href{https://www.lydsy.com/JudgeOnline/problem.php?id=5424}{【bzoj 5424】烧桥计划}}

\subsection{题目大意}

给定一个长度为 $n$ 的序列 $a_1,a_2,\dots,a_n$ 和一个参数 $m$

你要从中删掉若干个位置 $p_1,p_2, \dots ,p_k  (1 \le p_1 < p_2 < \dots < p_k \le n)$,耗费 $\sum_{i=1}^{k}i \times a_{p_i}$ 的代价。

上一步会把序列分割成 $k + 1$ 段,对于剩下的每段求和,如果某一段的和 $sum > m$,则要额外支付 $sum$ 的代价

$k$ 是你任选的,求最小总代价,其中 $n \le 100000, 1000 \le a_i \le 2000$

\subsection{算法讨论}

考虑一个 $O(n^2)$ 的暴力,设 $f_{i,j}$ 表示考虑完前 $i$ 个,第 $i$ 个要删掉,且一共删了 $j$ 个的最小代价

那么有:

$$
f_{i,j}=\min(f_{k,j-1}+\text{cost}(s_{i-1}-s_{k}))+j \times a_i
$$

其中 $\text{cost}(x)=x \times [x > m]$

转移的话可以考虑 $\text{cost}(s_{i-1}-s_{k})$ 是否为 $0$ 作为分界点

假设有一个 $k$,满足:$\forall 1 \le t \le k, s_{i-1}-s_{t} > m$,有转移:

$$
f_{i,j} = \min\{ f_{t,j-1}-s_t \} +j \times a_i+s_{i-1}
$$

同时这个 $k$ 应满足 $\forall k+1 \le t \le i-1, s_{i-1}-s_{t} \le m$,有转移:

$$
f_{i,j} = \min \{ f_{t,j-1} \} +j \times a_i
$$

对于前半部分,可以通过维护前缀最小值实现,对于后半部分,可以通过维护单调队列来实现

设 $k$ 表示删了多少个,对于 $k=0$ 的初始解,它的至多代价为 $2000n$

对于一个任意 $k$ 的解,它的至少代价为 $\sum_{i=1}^{k}1000=500k(k+1)$

那么如果一个 $k$ 可能会对答案产生更优的影响,则至少有 $500k(k+1) \le 2000n \Rightarrow k = O(\sqrt n)$

于是只需要保存 $k \le O(\sqrt n)$ 的解,时间复杂度将为 $O(n \sqrt n)$

\subsection{时间复杂度}

$$
O(n \sqrt n)
$$

\subsection{空间复杂度}

$$
O(n \sqrt n)
$$

\section{\href{https://www.lydsy.com/JudgeOnline/problem.php?id=2368}{【Google Code Jam 2008 APAC Onsites】Modern Art Plagiarism}}

\subsection{题目大意}

给定两棵无根树 $A$ 和 $B$,判断是否存在 $A$ 的一个子连通块和 $B$ 同构,其中树的节点数不超过 $100$

\subsection{算法讨论}

首先把无根树变为有根树再做,钦定 $B$ 的根为 $1$,然后枚举 $A$ 的根,使得 $A$ 和 $B$ 的根是同构的

如何判断两个棵树 $A_r,B_r$ 是否同构呢?

建立二分图,如果 $A_r$ 的某个儿子 $u$ 和 $B_r$ 的某个儿子 $v$ 同构,那么连一条从 $u$ 到 $v$ 的边,如果这个二分图有完美匹配,那么当前这个配对节点可以同构

\subsection{时间复杂度}

$$
O(n^2 \sqrt{n} n^2)=O(n^{4.5})
$$

\subsection{空间复杂度}

$$
O(n^2)
$$

\section{\href{https://www.zhixincode.com/contest/20/problem/J?problem_id=305}{【CCPC-Wannafly Winter Camp Day5 (Div1, onsite)】Special Judge}}

\subsection{题目大意}

有一个 $n(1 \le n \le 1000)$ 个点 $m(1 \le m \le 2000)$ 条边的图画在了平面上,你想知道有多少对边之间对应的线段相交

特别地,对于图中的一对边,如果有公共点且只在对应的端点相交,那么我们不认为这对边相交

\subsection{算法讨论}

大分类讨论题

先判断是否一条线段的两个端点都在另一条线段上,之后跨立实验来判断是否可能相交,然后特判两次跨立实验的面积都为 $0$,然后判断是否只在某个端点处有交点

\subsection{时间复杂度}

$$
O(m^2)
$$

\subsection{空间复杂度}

$$
O(n+m)
$$

\section{\href{https://www.zhixincode.com/contest/20/problem/E?problem_id=300}{【CCPC-Wannafly Winter Camp Day5 (Div1, onsite)】Fast Kronecker Transform}}

\subsection{题目大意}

给定两个序列 $a_0,a_1, \cdots, a_n$ 和 $b_0,b_1, \cdots, b_m$,求一个序列 $c_0,c_1, \cdots, c_{n+m}$,满足:

$$
c_{k}=\sum_{i+j=k}ij \times [a_i=b_j]
$$

\subsection{算法讨论}

枚举权值,设出现次数为 $x$

如果 $x \le T$,那么可以直接 $O(x^2)$ 暴力卷积

否则把出现该权值的位置标为对应下标,其它位置为 $0$,然后进行 $NTT$,这个部分的时间复杂度为 $O(x \log x)$

综上,总的时间复杂度为 $O(T^2 \frac{n}{T} + n \log n \frac{n}{T})=O(nT+n \log n \frac{n}{T})$

当 $nT=n \log n \frac{n}{T}$ 时,即 $T=\sqrt{n \log n}$,时间复杂度为 $O(n \sqrt {n \log n})$

实际上,由于种种原因,应该令 $T=10^4$

\subsection{时间复杂度}

$$
O(n \sqrt{n \log n})
$$

\subsection{空间复杂度}

$$
O(n)
$$

\section{\href{https://www.zhixincode.com/contest/20/problem/I?problem_id=304}{【CCPC-Wannafly Winter Camp Day5 (Div1, onsite)】Sorting}}

\subsection{题目大意}

你有一个数列 $a_1, a_2, \dots, a_n$,你要模拟一个类似于快速排序的过程,同时给定一个固定的数字 $x$

一共有 $q$ 次操作,诸如如下三种:

1. 询问区间 $[l, r]$ 之间的元素的和,也就是 $\sum_{i=l}^r a_i$

2. 对区间 $[l,r]$ 进行操作,也就是说你把区间中所有的数字拿出来,然后把小于等于 $x$ 的数字按顺序放在左边,把大于 $x$ 的数字按顺序放在右边,把这些数字接起来,放回到数列中

3. 对区间 $[l,r]$ 进行操作,也就是说你把区间中所有的数字拿出来,然后把大于 $x$ 的数字按顺序放在左边,把小于等于 $x$ 的数字按顺序放在右边,把这些数字接起来,放回到数列中

其中 $1 \le n,q \le 2 \times 10^5, 0 \le x \le 10^9, 1 \le a_i \le 10^9$

\subsection{算法讨论}

这道题的主要难点是在于读清楚题意,这个 $x$ 是一个常数

那么就好做了,把 $a_i \le x$ 的那一些标为 $1$,把 $a_i > x$ 的那些标为 $0$,然后后两个操作相当于区间 $01$ 排序,直接线段树维护区间赋值即可

由于是按照顺序重新排列,因此所有标为 $0$ 的数字,它们在原先序列上的相对顺序不变,$1$ 同理

然后在查询的时候只需要分别知道查询区间的 $01$ 的个数,和之前的 $01$ 的个数即可

\subsection{时间复杂度}

$$
O((n+q) \log n)
$$

\subsection{空间复杂度}

$$
O(n)
$$

\section{\href{https://code.mi.com/problem/list/view?id=125}{【小米 OJ 编程比赛 01 月常规赛】灯}}

\subsection{题目大意}

一个屋子有 $n$ 个开关控制着 $n$ 盏灯,但奇怪的是,每个开关对应的不是一盏灯,而是 $n-1$ 盏灯

每次按下这个开关,其对应的 $n-1$ 盏灯就会由亮变灭,或者由灭变亮

保证不会有两个开关控制同样的 $n-1$ 盏灯

现在刘同学想把灯全部开好,但是这些灯一开始的状态非常乱,刘同学想知道最少需要按多少次开关才能使所有灯全部亮起

\subsection{算法讨论}

这出题人水平不行啊,抄原题就算了,题面写的还不清楚

通过猜想题意,可以假设题意是这样的:有 $n$ 个灯泡和 $n$ 个开关,一开始编号为 $1 \sim l$ 的灯泡是亮的,第 $i$ 个开关按下后会让所有除了第 $i$ 号灯泡外的其它灯泡全部翻转,求最少按多少次开关可以使得所有灯全部亮起

按照套路,设 $x_i$ 表示第 $i$ 个开关是否按下,显然 $x_i \in \{0,1\}$,且答案就是 $\sum_{i=1}^{n}x_i$ 的最小值

同时为了满足灯泡最后都是亮的这个要求,则有如下约束:

$$
\begin{cases}
\oplus_{j=1 \wedge i \not= j}^{n}x_j = 0 \quad & (1 \le i \le l) \\
\oplus_{j=1 \wedge i \not= j}^{n}x_j = 1 \quad & (l+1 \le i \le n)
\end{cases}
$$

显然,只需要枚举 $T=\oplus_{i=1}^{n}x_i$ 的值,就会得到这个方程组的唯一解

如果 $T=0$,那么 $\forall 1 \le i \le l,x_i=0$,且 $\forall l+1 \le i \le n,x_i=1$

如果 $T=1$,那么 $\forall 1 \le i \le l,x_i=1$,且 $\forall l+1 \le i \le n,x_i=0$

\subsection{时间复杂度}

$$
O(1)
$$

\subsection{空间复杂度}

$$
O(1)
$$

\section{\href{https://www.zhixincode.com/contest/28/problem/E?problem_id=408}{【CCPC-Wannafly Winter Camp Day8 (Div1, onsite)】Souls-like Game}}

\subsection{题目大意}

给定 $n$ 个 $3 \times 3$ 的矩阵,有 $m$ 次操作,每次诸如把 $[l,r]$ 中所有的矩阵都变为输入的一个矩阵,或者查询 $[l,r]$ 的所有矩阵的连乘积后的所有元素的和

输出对 $998244353$ 取模,且 $n,m \le 10^5$

\subsection{算法讨论}

如果直接进行线段树维护矩阵,时间复杂度是 $O(26 n\log^2n)$ 的,无法通过此题

考虑先把 $n$ 扩充到 $2^k$,这样对于每次修改,只会有 $O(\log n)$ 个参与修改的矩阵,于是可以先对于所有修改,都预处理出来,然后进行修改

\subsection{时间复杂度}

$$
O(27 (n+q) \log n)
$$

\subsection{空间复杂度}

$$
O(27 (n+q \log n))
$$

\section{\href{https://www.zhixincode.com/contest/28/problem/I?problem_id=412}{【CCPC-Wannafly Winter Camp Day8 (Div1, onsite)】岸边露伴的人生经验}}

\subsection{题目大意}

给定 $n$ 个 $10$ 维向量 $\{V_n\}$,每一维的值在 $\{0,1,2\}$ 中

定义向量 $\overrightarrow{V}=(x_1,x_2, \dots ,x_10)$ 的模长 $|\overrightarrow{V}|$ 为 $\sqrt{x_1^2+x_2^2+x_3^2+\dots+x_{10}^2}$

求有多少个四元组 $(i,j,k,l)$,满足 $1 \le i,j,k,l \le n$ 且 $|\overrightarrow{V_i}-\overrightarrow{V_j}|=|\overrightarrow{V_k}-\overrightarrow{V_l}|$

\subsection{算法讨论}

首先去掉根号后答案不变,为了方便起见,所有的模长都平方一下

之后相当于对于所有 $0 \le x \le 40$,求有多少 $(i,j)$,满足 $1 \le i,j \le n$,且 $|\overrightarrow{V_i}-\overrightarrow{V_j}|=x$,求完之后对于所有 $x$,把对数平方再求和就是答案

考虑如下的一张差的平方的真值表:

$$
\begin{array} {c|lcr}
 & \text{0} & \text{1} & \text{2} \\
\hline
0 & 0 & 1 & 4 \\
1 & 1 & 0 & 1 \\
2 & 4 & 1 & 0
\end{array}
$$

再考虑异或的真值表:

$$
\begin{array} {c|lcr}
 & \text{0} & \text{1} & \text{2} \\
\hline
0 & 0 & 1 & 2 \\
1 & 1 & 0 & 3 \\
2 & 2 & 3 & 0
\end{array}
$$

可以得到一个异或向差方的映射:

$$
\begin{cases}
0 \Rightarrow 0 \\
1 \Rightarrow 1 \\
2 \Rightarrow 4 \\
3 \Rightarrow 1
\end{cases}
$$

考虑把向量 $\overrightarrow{V}$ 看成 $10$ 个 $4$ 进制数,然后求异或卷积,即可得到一个异或数组,然后依次映射到差方数组即可

\subsection{时间复杂度}

$$
O(2^{20} \times 20)
$$

\subsection{空间复杂度}

$$
O(2^{20})
$$

\section{\href{https://www.zhixincode.com/contest/28/problem/G?problem_id=410}{【CCPC-Wannafly Winter Camp Day8 (Div1, onsite)】穗乃果的考试}}

\subsection{题目大意}

给定一个 $n \times m$ 的 $01$ 矩阵,设 $f_i$ 表示恰好有 $i$ 个 $1$ 的子矩阵个数,求:

$$
\sum_{i=0}^{nm}i^2f_i
$$


\subsection{算法讨论}

考虑 $x^2$ 的等价变形:

$$
x^2=\sum_{i=1}^{x}\sum_{j=1}^{x}1 \times 1
$$

于是可以枚举两个为 $1$ 的格子,然后计算有多少个子矩阵包括它们

通过枚举其中一个格子后,另一个格子通过前缀和来计算

\subsection{时间复杂度}

$$
O(nm)
$$


\subsection{空间复杂度}

$$
O(nm)
$$

\section{\href{https://www.zhixincode.com/contest/24/problem/J?problem_id=358}{【CCPC-Wannafly Winter Camp Day7 (Div1, onsite)】强壮的排列}}

\subsection{题目大意}

给定奇数 $n$,求有多少个 $1 \sim n$ 的全排列,满足 $\forall 2 \le i \le n$,若 $2 \mid i$,则 $a_{i-1} < a_i$,否则 $a_{i-1} > a_i$

其中 $1 \le n \le 10^5$,答案对 $998244353$ 取模

\subsection{算法讨论}

设 $a_n$ 表示 $1 \sim n$ 的满足条件的全排列的个数,则:

$$
\begin{cases}
a_0=1 \\
a_1=1 \\
a_n=\sum_{i=0}^{n-1} {n - 1 \choose i} a_i a_{n-1-i} [2 \not| i] \quad (n \ge 1)
\end{cases}
$$

意义就是考虑第 $n$ 个数是必须放到奇数位置上,也就是说 $n$ 左侧必须要有奇数个数字

由于输入的 $n$ 是奇数,所以可以强行钦定 $a_{2x}=0$,即所有偶数项都为 $0$,然后就可以去掉 $[2 \not| i]$ 这一项了,即:

$$
\begin{cases}
a_0=0 \\
a_1=1 \\
a_n=\sum_{i=0}^{n-1}{n-1 \choose i} a_i a_{n-1-i} \quad (n \ge 1)
\end{cases}
$$

整理可得:

$$
\begin{aligned}
&a_n=\sum_{i=0}^{n-1}{n-1 \choose i} a_i a_{n-1-i} \\
\Rightarrow &a_n=\sum_{i=0}^{n-1}\frac{(n-1)!}{i!(n-1-i)!}a_ia_{n-1-i} \\
\Rightarrow &n\frac{a_n}{n!}=\sum_{i=0}^{n-1}\frac{a_i}{i!}\frac{a_{n-1-i}}{(n-1-i)!} \\
\Rightarrow &nf_n=\sum_{i=0}^{n-1}f_if_{n-1-i}
\end{aligned}
$$

设 $F(x)$ 表示 $\{f_n\}$ 的生成函数,则有:

$$
\begin{aligned}
F'(x)=F^2(x)+1 \\
\end{aligned}
$$

设 $y=F(x)$,则:

$$
\begin{aligned}
& \frac{dy}{dx}=y^2+1 \\
\Rightarrow & \int \frac{dy}{y^2+1}=\int 1 dx \\
\Rightarrow & \arctan(y)=x+C \\
\Rightarrow & y=\tan(x+C)
\end{aligned}
$$

令 $C=0$,所以:

$$
\begin{aligned}
F(x)=\tan(x)=\frac{\sin(x)}{\cos(x)}
\end{aligned}
$$

泰勒展开后多项式求逆即可

补充一个积分的推导:

设 $x=\tan y$,即 $y=\arctan x$

$$
\begin{aligned}
&\int \frac{1}{x^2+1}dx \\
=&\int \frac{1}{\tan^2y+1}d \tan y \\
=&\int \frac{1}{\frac{\sin^2y}{\cos^2y}+1}d \frac{\sin y }{\cos y} \\
=&\int \frac{\cos^2y}{\sin^2y+\cos^2y} \frac{\cos^2y-(-\sin^2y)}{\cos^2y}dy\\
=&y+C \\
=&C+\arctan x
\end{aligned}
$$

\subsection{时间复杂度}

$$
O(n \log n)
$$

\subsection{空间复杂度}

$$
O(n)
$$

\section{\href{https://www.zhixincode.com/contest/28/problem/B?problem_id=405}{【CCPC-Wannafly Winter Camp Day8 (Div1, onsite)】玖凛两开花}}

\subsection{题目大意}

给出一张点集为 $V$,边集为 $E$ 的无向图 $G$,点的编号为 $0$ 至 $|V|-1$,边 $(u,v)$ 的权值为 $\min(u,v)$

一个边集 $S$ 是图的一个匹配当且仅当 $S \subseteq E$,且 $\forall e_1,e_2 \in S \bigwedge e_1 \neq e_2$,满足 $e_1,e_2$ 无公共端点

对于一个边集 $S$,定义 $W_S$ 为 $S$ 中所有边的权值的集合

对于一个自然数集 $W$,定义 $Mex(W)$ 为最小的不属于 $W$ 的自然数

求对于图 $G$ 的匹配 $S$,$Mex(W_S)$ 的最大值是多少

好心的 $Rinne$ 为了减少你的负担,将题目的做法告诉了你,你只需要实现一个高效的开花算法即可

当然,如果你已经会做这道题了,就可以不用继续看下去了

$Rinne$ 给出的做法是这样的:

对于所有的边 $e \in E$,若其原本的边权为 $w$,将其改为 $2^{|V|-w}$ 

求出新图的最大权匹配后,设其权值之和为 $X$,将其二进制表示中的最低 $|V|+1$ 位由高位到低位依次写出来,第一个为 $0$ 的位的出现位置(从 $0$ 开始编号)就是答案

其中 $1 \le |V| \le 10^4, 1 \le |E| \le 2 \times 10^4$

\subsection{算法讨论}

二分答案后,可以把原图拆分为两个二分图,只需判断是否完全匹配即可

\subsection{时间复杂度}

$$
O(m \sqrt n \log n)
$$

\subsection{空间复杂度}

$$
O(n)
$$

\section{\href{https://www.lydsy.com/JudgeOnline/problem.php?id=1566}{【NOI2009】管道取珠}}

\subsection{题目大意}

管道取珠是小 $X$ 很喜欢的一款游戏。在本题中,我们将考虑该游戏的一个简单改版

游戏初始时,左侧上下两个管道分别有一定数量的小球(有深色球和浅色球两种类型),而右侧输出管道为空

每一次操作,可以从左侧选择一个管道,并将该管道中最右侧的球推入右边输出管道

假设上管道中有 $n$ 个球, 下管道中有 $m$ 个球,则整个游戏过程需要进行 $n+m$ 次操作,即将所有左侧管道中的球移入输出管道

最终 $n+m$ 个球在输出管道中从右到左形成输出序列

爱好数学的小 $X$ 知道,他共有 ${n+m \choose n}$ 种不同的操作方式,而不同的操作方式可能导致相同的输出序列

假设最终可能产生的不同种类的输出序列共有 $K$ 种,其中:第 $i$ 种输出序列的产生方式(即不同的操作方式数目)有 $a_i$ 个

聪明的小 $X$ 早已知道,$\sum a_i={n+m \choose n}$

因此,小X希望计算得到:$\sum a_i^2$,由于这个值可能很大,因此只需要输出该值对 $1024523$ 的取模即可

其中 $n, m \le 500$

\subsection{算法讨论}

题意相当于把所有的可能的操作放到一个 $map$ 中,其中相同的判定为生成的序列相同,然后输出所有相同集的大小的平方和

如果把序列复制一倍,相当于公共子序列的个数,因为在平方的时候,有序点对 $(i,j)$ 会被计算一次

于是可以设 $f_{i,j,k,l}$ 分别表示原序列操作到了 $(i,j)$,复制序列操作到了 $(k,l)$ 时的公共子序列个数

由于 $k+l=i+j$,所以可以移除掉 $l$ 这一维,然后 $i$ 这一维可以滚动数组优化掉

\subsection{时间复杂度}

$$
O(n^3)
$$

\subsection{空间复杂度}

$$
O(n^2)
$$

\section{\href{https://www.lydsy.com/JudgeOnline/problem.php?id=1108}{【POI 2007】天然气管道}}

\subsection{题目大意}

平面上分别有 $n(n \le 50000)$ 个黑点和白点,每个黑点只能像在它右下方的白点匹配

求一个黑白完美匹配,使得匹配的点对的曼哈顿距离和最小

\subsection{算法讨论}

若一个黑点为 $(x,y)$,那么它对答案的贡献就是 $y-x$,白点为 $x-y$

\subsection{时间复杂度}

$$
O(n)
$$

\subsection{空间复杂度}

$$
O(1)
$$

\section{\href{http://acm.hdu.edu.cn/showproblem.php?pid=5887}{【2016 ACM/ICPC Asia Regional Qingdao Online
】Herbs Gathering}}

\subsection{题目大意}

给定 $n(n \le 100)$ 个物品,每个物品有 $(w,v)$ 两个属性,要求从中选出一些物品,使得 $\sum w$ 不超过 $m(m \le 10^9)$,最大化 $\sum v$,保证数据随机生成

\subsection{算法讨论}

假设存在一个状态 $(\sum w,\sum v)$,如果存在另一个状态 $(\sum w',\sum v')$,其中 $\sum w' \le \sum w$ 且 $\sum v' \ge \sum v$,那么状态 $(\sum w, \sum v)$ 就没有存在的必要了

可以证明这么做的话单次的期望状态数为 $O(\log (n!))=O(n \log n)$

\subsection{时间复杂度}

$$
O(n^2 \log n)
$$

\subsection{空间复杂度}

$$
O(n \log n)
$$

\section{\href{https://nanti.jisuanke.com/t/34061}{【2018-2019 ICPC, Asia Xuzhou Regional Contest】Rikka with Subsequences}}

\subsection{题目大意}

给定一个长度为 $n(n \le 200)$ 的序列 $q$,每个位置是 $[1, n]$ 之间的整数

给定 $n \times n$ 的 $01$ 矩阵 $g$,定义一个序列 $a_1,a_2, \cdots, a_m$ 是好的,当且仅当且对于任意的 $1 \le i \le m$,有 $g_{a_i,a_{i+1}}=1$ 恒成立

假设有一个 $std::map$ 保存了 $q$ 的每个好的子序列的出现次数,你需要统计它们的出现次数的立方和

\subsection{算法讨论}

首先对于 $x^3$ 有一个等价变换:

$$
x^3=\sum_{i=1}^{x}\sum_{j=1}^{x}\sum_{k=1}^{x}1 \times 1 \times 1
$$

于是可以把序列 $q$ 复制成三份 $a,b,c$,然后求有多少个好的公共子序列

考虑第一个暴力,$f_{i,j,k}$ 表示第一个序列到了 $i$,第二个序列到了 $j$,第三个序列到了 $k$,且强制选上 $a_i,b_j,c_k$ 的好的公共子序列个数

转移即枚举 $i \to x, j \to y, k \to z, g_{a_i, a_{x}}=1,a_x=b_y=c_z$,时间复杂度为 $O(n^6)$

考虑优化这个暴力,将移动分为三个阶段:移动 $i$,移动 $j$,移动 $k$

于是可以设 $f_{i,j,k,0/1/2}$ 表示当前移动 $i/j/k$ 的方案数,若移动 $i$,则找一个 $x$,使得 $g_{a_i,a_x}=1$ 就行,若移动 $j$,则找一个 $y$ 使得 $a_i=b_y$ 就行,$k$ 和 $j$ 同理,此时时间复杂度为 $O(n^4)$

其实从枚举 $x$ 也是没有必要的

简化转移:要么将 $x$ 增加 $1$,要么钦定这个 $x$ 就是我们想要的 $x$,开始 $y$ 的转移

\subsection{时间复杂度}

$$
O(n^3)
$$

\subsection{空间复杂度}

$$
O(n^3)
$$

\section{\href{https://vjudge.net/problem/URAL-2057}{【ural 2057】Non-palindromic cutting}}

\subsection{题目大意}

给定一个长度为 $n$ 的字符串 $S$

将 $S$ 划分为若干段非空连续子串,使得每段都不是回文串

求最多能划分成多少段

\subsection{算法讨论}

设 $f_i$ 表示 $s_{1 \dots i}$ 的最大划分数,则有:

$$
f_i=\max(f_j)+1
$$

其中 $s_{j+1 \dots i}$ 是一个有解的字符串,即不是 $aaaaa, aabaa, ababa$ 的情况

\subsection{时间复杂度}

$$
O(n)
$$

\subsection{空间复杂度}

$$
O(n)
$$

\section{\href{https://www.lydsy.com/JudgeOnline/problem.php?id=2817}{【ZJOI 2012】波浪}}

\subsection{题目大意}

考虑 $1 \sim n$ 的一个排列 $a_1,a_2,\dots,a_n$,定义它的波浪值为:

$$
\sum_{i=1}^{n-1} |a_i-a_{i+1}|
$$

给定 $m$,求有多少排列的波浪值不小于 $m$,其中 $n \le 100$

\subsection{算法讨论}

考虑从小到大的顺序依次插入 $1 \sim n$,设 $f_{i,j,k,l}$ 表示现在要插入 $i$,已经有了 $j$ 个极大连续块,此时的贡献和为 $k$,有 $l$ 个边界被碰触了

对于绝对值符号,可以拆为:

$$
|a-b|=\max(a,b)-\min(a,b)
$$

考虑如下几种转移:

\textbf{1. 放在两侧,不和任何已有的极大连续块相邻}

此时的方案数为 $2-l$,且有贡献为 $-i$

\textbf{2. 放在两侧,和某个极大连续块相邻}

此时的方案数为 $2-l$,且有贡献为 $i$

\textbf{3. 合并两个已存在的极大连续块}

此时的方案数为 $j-1$,且有贡献为 $2i$

\textbf{4. 独自成为一个极大连续块}

此时方案数为 $j+1-l$,且有贡献 $-2i$

\textbf{5. 只和一个已存在的极大连续块相邻}

此时方案数为 $2j-l$,且有贡献 $i-i=0$

\subsection{时间复杂度}

$$
O(n^4)
$$

\subsection{空间复杂度}

$$
O(n^3)
$$

\section{\href{http://www.51nod.com/Challenge/Problem.html?problemId=1594}{【51nod 1594】Gcd and Phi}}

\subsection{题目大意}

给定 $n(n \le 2 \times 10^6)$,求:

$$
\sum_{i=1}^{n}\sum_{j=1}^{n}\phi(\gcd(\phi(i),\phi(j)))
$$

\subsection{算法讨论}

设 $f(d)$ 为:

$$
f(d)=\sum_{i=1}^{n}\sum_{j=1}^{n}[d | \phi(i)][d | \phi(j)]=\left(\sum_{i=1}^{n}[d | \phi(i)]\right)^2
$$

设 $g(d)$ 为:

$$
g(d)=\sum_{i=1}^{n}\sum_{j=1}^{n}[d = \gcd(\phi(i), \phi(j))]
$$

则有:

$$
g = f \times \mu
$$

同时答案为:

$$
\sum_{i=1}^{n}\phi(i)g(i)
$$

\subsection{时间复杂度}

$$
O(n \log n)
$$

\subsection{空间复杂度}

$$
O(n)
$$

\section{\href{http://www.51nod.com/Challenge/Problem.html?problemId=1301}{【51nod 1301】集合异或和}}

\subsection{题目大意}

给定 $n,m(1 \le n,m \le 2000)$,构造两个不可重整数集 $X,Y$,满足 $\oplus X < \oplus Y$,且 $\forall x \in X,1 \le x \le n$,且 $\forall y \in Y,1 \le y \le n$

求有多少种构造方案,两种方案不同当且仅当 $(X,Y)$ 至少一个不同

\subsection{算法讨论}

考虑 $a<b$ 的条件,即二进制下有一段相同的前缀,然后在某一位处 $a$ 为 $0$,且 $b$ 为 $1$

那么可以枚举这一位 $l$,然后设 $f_{i, j, t}$ 表示考虑完 $1 \sim i$,此时 $(\oplus X) \oplus (\oplus Y)=j$,同时 $\oplus Y$ 的二进制下第 $l$ 位为 $t$ 的方案数

那么如果 $\oplus X$ 和 $\oplus Y$ 在第 $l$ 位不同,那么方案数为:

$$
\sum_{i=2^l}^{2^{l+1}-1}f_{\max(n,m), i, 1}
$$

\subsection{时间复杂度}

$$
O(n^2 \log n)
$$

\subsection{空间复杂度}

$$
O(n)
$$

\section{\href{http://www.51nod.com/Challenge/Problem.html?problemId=1577}{【51nod 1577】异或凑数}}

\subsection{题目大意}

给定 $n(n \le 5 \times 10^5)$ 个整数 $a_1 \sim a_n(0 < a_i < 2^{30})$

有 $m(m \le 5 \times 10^5)$ 次询问,每次给定 $l, r, k(0 < k < 2^{30})$,判断是否可以从 $a_l \sim a_r$ 中选择一些数,使得它们的异或和为 $k$

\subsection{算法讨论}

枚举右端点,考虑维护 $w$ 个线性无关的数,使得从高位往低位这些数的存在位置尽可能靠右

\subsection{时间复杂度}

$$
O(n \log a_i)
$$

\subsection{空间复杂度}

$$
O(n \log a_i)
$$

\section{\href{http://www.51nod.com/Challenge/Problem.html?problemId=1312}{【51nod 1312】最大异或和}}

\subsection{题目大意}

给定 $n$ 个整数 $a_1 \sim a_n(0 \le a_i \le 10^{15})$,每次可以选择一对 $(i,j)$,且 $i \not= j$,设 $x=a_i \oplus a_j, y=a_j$,之后让 $a_i=x,a_j=y$

可以随便操作,最大化 $\sum_{i=1}^{n}a_i$ 的值

\subsection{算法讨论}

考虑提取这个集合的一个线性基,则除了这个线性基外,其它整数一定可以变为这个空间下的最大值,线性基中的某一个也可以

于是只需要最小化这个线性基的和即可,直接高斯消元求出来的线性基可以保证对于每一个 $1$ 的位,是取得该位置的最小值

还有另外一种求法,就是先求出线性基后,每个线性基异或上所有位数比它小的那些

\subsection{时间复杂度}

$$
O(n \log a_i)
$$

\subsection{空间复杂度}

$$
O(\log a_i)
$$


\section{\href{http://www.51nod.com/Challenge/Problem.html?problemId=1479}{【51nod 1479】小Y的数论题}}
\subsection{题目大意}

给定三个两两互质的数 $a, b, c(1 \le a, b, c \le 10^9)$,以及另一个数 $m(3 \le m \le 10^9)$

求三个 $(0, m)$ 范围内的整数 $x, y, z$,使得 $x^a+y^b \equiv z^c \pmod {m}$

\subsection{算法讨论}

如果 $m = 2^k$,则可以分类讨论:

\textbf{1. a=b=c=1}

$$
\begin{cases}
x=1 \\
y=1 \\
z=2
\end{cases}
$$

\textbf{2. a>1}

$$
\begin{cases}
x=\frac{m}{2} \\
y=1 \\
z=1
\end{cases}
$$

\textbf{3. b>1}

$$
\begin{cases}
x=1 \\
y=\frac{m}{2} \\
z=1
\end{cases}
$$

\textbf{4. c>1}

$$
\begin{cases}
x=\frac{m}{2} \\
y=\frac{m}{2} \\
z=\frac{m}{2}
\end{cases}
$$

如果 $m \not= 2^k$,则有:

$$
\begin{cases}
x=2^{bk} \\
y=2^{ak} \\
z=2^{l}
\end{cases}
$$

可以推出:$abk+1=cl$,只需要 $exgcd$ 解出一组即可

\subsection{时间复杂度}

$$
O(\log m)
$$

\subsection{空间复杂度}

$$
O(1)
$$

\section{\href{https://www.lydsy.com/JudgeOnline/problem.php?id=2844}{【bzoj 2844】albus就是要第一个出场}}

\subsection{题目大意}

给定一个可重集 $S$,将它所有子集的异或和都求出来后,从小到大排序,求数字 $x$ 的排名

其中 $|S| \le 10^5$,所有元素的值不超过 $10^9$

\subsection{算法讨论}

设 $T$ 为这个异或空间内所能组成的数字的集合,设 $f(x)$ 表示这个异或空间组成 $x$ 的方案数

则有 $f(x)=2^{n-|B|}$,其中 $B$ 表示线性基的大小

于是可以从低位往高位贪心,算出小于 $x$ 的不同数字个数,然后乘以 $2^{n-|B|}$ 后加一就是 $x$ 的排名

\subsection{时间复杂度}

$$
O(\log x)
$$

\subsection{空间复杂度}

$$
O(\log x)
$$

\section{\href{https://www.lydsy.com/JudgeOnline/problem.php?id=2460}{【BJWC 2011】元素}}

\subsection{题目大意}

给定 $n(1 \le n \le 1000)$ 个物品,每个物品有两个属性 $(id,w)(id \le 10^{18},w \le 10^4)$,要求选出一些物品,使得 $\oplus id \not= 0$,最大化 $\sum w$

\subsection{算法讨论}

按照 $w$ 从大到小排序后,判断每次是否可以插入当前的 $id$,如果可以插入就累加入答案

\subsection{时间复杂度}

$$
O(n (\log n + \log id))
$$

\subsection{空间复杂度}

$$
O(\log id)
$$

\section{\href{https://www.lydsy.com/JudgeOnline/problem.php?id=3105}{【CQOI 2013】新Nim游戏}}

\subsection{题目大意}

传统的 $Nim$ 游戏是这样的:有一些火柴堆,每堆都有若干根火柴(不同堆的火柴数量可以不同),两个游戏者轮流操作,每次可以选一个火柴堆拿走若干根火柴,可以只拿一根,也可以拿走整堆火柴,但不能同时从超过一堆火柴中拿,拿走最后一根火柴的游戏者胜利

本题的游戏稍微有些不同:在第一个回合中,第一个游戏者可以直接拿走若干个整堆的火柴,可以一堆都不拿,但不可以全部拿走,第二回合也一样,第二个游戏者也有这样一次机会,从第三个回合(又轮到第一个游戏者)开始,规则和 $Nim$ 游戏一样

如果你先拿,怎样才能保证获胜?如果可以获胜的话,还要让第一回合拿的火柴总数尽量小

火柴堆数不超过 $100$,每堆的火柴个数不超过 $10^9$

\subsection{算法讨论}

为了先手必胜,那么剩下的火柴一定可以全部插入线性基,否则如果有剩余的,则这个数字在插入的时候会变为 $0$,后手只需要保留这些变为 $0$ 的即可

为了使得总数最小,于是可以从大到小排序,依次插入

\subsection{时间复杂度}

$$
O(n (\log n + \log a_i))
$$

\subsection{空间复杂度}

$$
O(n+\log a_i)
$$

\section{\href{https://www.lydsy.com/JudgeOnline/problem.php?id=4347}{【POI 2016】Nim z utrudnieniem
}}

\subsection{题目大意}

$A$ 和 $B$ 两个人玩游戏,一共有 $m$ 颗石子,$A$ 把它们分成了 $n$ 堆,每堆石子数分别为 $a_1,a_2 \dots, a_n$

每轮可以选择一堆石子,取掉任意颗石子,但不能不取,谁先不能操作,谁就输了

在游戏开始前,$B$ 可以扔掉若干堆石子,但是必须保证扔掉的堆数是 $d$ 的倍数,且不能扔掉所有石子

$A$ 先手,请问 $B$ 有多少种扔的方式,使得 $B$ 能够获胜,输出对 $10^9+7$ 取模

其中 $1 \le n \le 5 \times 10^5, 1 \le d \le 10,1 \le a_i \le 10^6, \sum a_i \le 10^7$

\subsection{算法讨论}

设 $f_{i, j, k}$ 表示处理完前 $i$ 堆石子,选择了 $j \bmod d$ 堆,选择的石子的异或和为 $k$ 的方案数,那么答案为 $f_{n,0,\oplus a_i}-[d | n]$

考虑到 $a \oplus b \le 2 \max(a,b)$,于是可以先从小到大排序,然后依次考虑每一个 $a_i$,此时它只会更新不超过 $2 \times a_i$ 个位置上的值,于是时间复杂度为 $O(md)$

\subsection{时间复杂度}

$$
O(md)
$$

\subsection{空间复杂度}

$$
O(md)
$$

\section{\href{http://acm.hdu.edu.cn/showproblem.php?pid=6166}{【hdu 6166】Senior Pan}}

\subsection{题目大意}

给定一张有非负边权的有向图 $G(2 \le |G| \le 10^5)$,以及一个关键点集 $K(2 \le |K| \le n)$,最小化 $\text{dis}(u,v)$,其中 $u,v \in K$ 且 $u \not= v$

\subsection{算法讨论}

考虑一个弱化版的问题:给定两个点集 $A,B$,最小化 $\text{dis}(u,v)$,其中 $u \in A,v \in B$

做法就是新建两个点 $S,T$,把 $S$ 连向所有 $A$ 中的点,同时把所有 $B$ 中的点连向 $T$,之后跑 $S$ 到 $T$ 的最短路

那么这道题也可以同样的做法,按照二进制的某一位是否为 $0$ 进行分类,然后跑这个弱化版的问题即可,显然任何一个关键点对都会至少有一次被分到两边

当然也可以直接随机分类,假设答案点对为 $(u,v)$,那么随机一次后会有 $0.5$ 的概率是分到同一边,多随机几次后至少一次分到两边的概率就很高了

\subsection{时间复杂度}

$$
O(n \log^2 n)
$$

\subsection{空间复杂度}

$$
O(n)
$$

\section{\href{https://codeforces.com/problemset/problem/724/G}{【codeforces 724G】Xor-matic Number of the Graph}}

\subsection{题目大意}

给定有非负边权的无向图 $G(|G| \le 10^5)$,如果一个三元组 $(u,v,w)$ 满足 $1 \le u < v \le n$,且存在一条从 $u$ 到 $v$ 的路径,使得依次经过的边权的异或和为 $w$,那么这个三元组是好的

求所有好的三元组的 $w$ 的值之和,答案模 $10^9+7$

\subsection{算法讨论}

按位考虑有多少个 $(u,v,w)$ 满足 $w$ 二进制下存在第 $k$ 位,然后方案数乘 $2^k$ 后累加就行了

具体做法就是 $dfs$ 后把所有环都放入线性基,然后考虑线性基中是否有一个有 $k$ 这一位

\subsection{时间复杂度}

$$
O(n \log w^2)
$$

\subsection{空间复杂度}

$$
O(\log w)
$$

\section{\href{https://codeforces.com/problemset/problem/939/D}{【codeforces 939D】Love Rescue}}

\subsection{题目大意}

给定两个字符串 $s,t(|s|=|t| \le 10^5$,每次可以选择两个字符 $c_1,c_2$,然后把所有的 $c_1$ 都变成 $c_2$

\subsection{算法讨论}

考虑模拟题意,从左往右依次枚举 $s_i,t_i$,如果不相等就说明要把它们合并到一块,用并查集维护这个过程

\subsection{时间复杂度}

$$
O(n)
$$

\subsection{空间复杂度}

$$
O(n)
$$

\section{\href{https://zhixincode.com/contest/24/problem/A?problem_id=349}{【CCPC-Wannafly Winter Camp Day7 (Div1, onsite)】迷宫}}

\subsection{题目大意}

有一个 $n(n \le 10^5)$ 个点 $n-1$ 条边的无向连通图迷宫,其中有些点上面有人

现在所有人的目标都是逃离这个迷宫,而迷宫的出口是 $1$ 号点,每一时刻,会依次发生以下的事情:

\begin{quotation}
1. 在点 $x$ 上的人选择一个点 $f(x)$ 作为目标,要求 $f(x)$ 必须是 $x$,或者与 $x$ 有边相连的点,且对于 $x\neq y$,有 $f(x)\neq f(y)$

2. 在点 $x$ 上的人移动到 $f(x)$

3. 在点 $1$ 号点上的人成功逃脱,从这个游戏里消失
\end{quotation}

现在你需要求的是:让所有人都成功逃脱至少需要多少时间

\subsection{算法讨论}

考虑将操作翻转,即所有人从 $1$ 开始往目标点行走

此时除了 $1$ 以外,其它点每时每刻至多只有 $1$ 个人

此时所有人不会在 $1$ 以外的点处停止,也就是每时每刻都在往目标点行走

记 $f(x)$ 表示目标距离大于等于 $x$ 的点的个数,那么答案就是:

$$
\max\{\text{dis}(i)+f(\text{dis}(i))-1\}
$$

意思就是,对于一个最后走的,距离为 $x$ 的人,他需要等待其他所有距离不比他近的人先走

\subsection{时间复杂度}

$$
O(n)
$$

\subsection{空间复杂度}

$$
O(n)
$$

\section{\href{https://zhixincode.com/contest/2/problem/C?problem_id=25}{【CCPC-Wannafly Winter Camp Day1 (Div1, onsite)】拆拆拆数}}

\subsection{题目大意}

读入 $A$ 和 $B$,需要把 $A$ 拆成 $a_1, a_2, \dots, a_n$,把 $B$ 拆成 $b_1, b_2, \dots, b_n$,满足:

\begin{quotation}
1. $\forall 1 \leq i \leq n,$,有 $a_i, b_i \geq 2$ 且 $gcd(a_i, b_i) = 1$

2. $\sum_{i=1}^{n}{a_i} = A, \sum_{i=1}^{n}{b_i} = B$
\end{quotation}

如果有多组满足条件的 $a$ 和 $b$,请输出 $n$ 最小的任意一组即可

如果无解,请输出 $−1$

\subsection{算法讨论}

首先如果 $\gcd(A,B)=1$,那么显然直接输出就行了

否则断言 $n=2$,且 $\min(a_1,a_2) \le C,\min(b_1,b_2) \le C$,其中 $C$ 为某一常数,断言 $C=10$

\subsection{时间复杂度}

$$
O(1)
$$

\subsection{空间复杂度}

$$
O(1)
$$

\section{\href{https://www.lydsy.com/JudgeOnline/problem.php?id=2616}{【bzoj 2616】SPOJ PERIODNI}}

\subsection{题目大意}

有 $n$ 个 $1 \times h_i$ 的矩形小棋盘,底边边长为 $1$,在一条直线上拼成了一个畸形的棋盘

高度 $h_i$ 给出,第 $i$ 个矩形的高度为 $h_i$

若两个车相互攻击仅当它们在同一列,或在同一行且在这一行它们之间棋盘格子都是存在的

现在要在这棋盘上放置恰好 $k$ 个互相不攻击的车,问有多少种方案

其中 $n \le 500,k \le 500,h_i \le 10^6$,答案模 $10^9+7$

\subsection{算法讨论}

建立笛卡尔树后,每个节点表示一个矩形,宽为 $h_i-h_{fa_i}$,长为该节点的子树大小

那么放置车的时候,无非是在子树中放一些,然后再在当前节点的矩形中放一些

在子树中放是:

$$
g_{u,i}=\sum_{k}f_{lc,k} f_{rc,j-k}
$$

然后处理当且节点中放一些车:

$$
f_{u,j}=\sum_{k} g_{u,j-k} {width-(j-k) \choose k} {height \choose k}k!
$$

\subsection{时间复杂度}

$$
O(n^3)
$$

\subsection{空间复杂度}

$$
O(n^2)
$$

\section{\href{https://zhixincode.com/contest/2/problem/H?problem_id=30}{【CCPC-Wannafly Winter Camp Day1 (Div1, onsite)】我爱割葱}}

\subsection{题目大意}

由于太肝了,$wls$ 现在很割葱

假设葱一共有 $n$ 棵,第 $i$ 棵葱的高度为 $a_i$

$wls$ 一共要割最多 $k$ 刀葱,每刀可以在某一高度割去连续一段葱

以高度 $h$ 在区间 $[l, r]$ 割一刀葱是合法的,当且仅当区间里的葱的高度都不小于 $h$

此时,这个区间中的葱小于等于 $h$ 的未被割的部分都会被割掉

下面的葱被割掉以后,上面的葱不会掉下来

请问,$k$ 刀以后,割掉的葱的总长度的最大值是多少?

其中 $1 \le n, k \le 100, 1 \le a_i \le 10^6$

\subsection{算法讨论}

一个比较直观的想法是,建立笛卡尔树后考虑树上 $dp$

设 $f_{i,j,k}$ 表示以 $i$ 为根的子树,一共割了 $j$ 刀,且下表面占了 $k$ 个格子时的最大收益

转移十分显然,但粗略分析时间复杂度的话是 $O(n^5)$ 的,如果转移的时候只枚举到了子树大小,那么时间复杂度就降为了 $O(n^3)$

\subsection{时间复杂度}

$$
O(n^3)
$$

\subsection{空间复杂度}

$$
O(n^3)
$$

\section{\href{https://vjudge.net/problem/Kattis-scaffolding}{【 ICPC 2016 Hong Kong】G. Scaffolding}}

\subsection{题目大意}

你要搭一个一共 $n(n \le 10^5)$ 列的脚手架,第 $i$ 列放 $h_i$ 根竹子。

你每次只能搬 $m$ 根竹子上脚手架,并且只能在脚手架左右和向上移动

同时如果左边右边和上面没有竹子,也可以放一根竹子在那里,问最少需要多少次搬完

\subsection{算法讨论}

考虑操作翻转过来,变为每次拆一些竹子,先把笛卡尔树搞出来

设 $f_u$ 表示拆完以 $u$ 为根的子树的最少次数,$g_u$ 表示在最少次数的前提下,还能顺便拆多少

转移就直接贪心的能拿就拿即可

\subsection{时间复杂度}

$$
O(n)
$$

\subsection{空间复杂度}

$$
O(n)
$$

\section{\href{http://www.51nod.com/Challenge/Problem.html?problemId=1766}{【51nod 1766】树上的最远点对}}

\subsection{题目大意}

$n$ 个点被 $n-1$ 条边连接成了一颗树,给出 $a \sim b$ 和 $c \sim d$ 两个区间,表示点的标号请你求出两个区间内各选一点之间的最大距离,即你需要求出 $\max\{\text{dis}(i,j) |a \le i \le b, c \le j \le d\}$

其中 $n \le 10^5$,树的边权都是正整数

\subsection{算法讨论}

首先树上的一个点集的最远点对的距离,等价于这个点集构成的虚树的直径,在边权都是正整数的情况下,对于两个点集 $A,B$,假设它们的某个最远点对分别为 $(p_A,q_A),(p_B,q_B)$,那么 $A \cup B$ 的最远点对在 $p_A,q_A,p_B,q_B$ 中任意选两个的组合之中

于是可以线段树维护区间的任意的一个可行点对,注意查询距离的时候会用到 $lca$,这个需要单次 $O(1)$ 查询

\subsection{时间复杂度}

$$
O(n \log n)
$$

\subsection{空间复杂度}

$$
O(n \log n)
$$

\section{\href{https://www.zhixincode.com/contest/8/problem/G?problem_id=128}{【CCPC-Wannafly Winter Camp Day2 (Div2, onsite)】Linear Congruential Generator}}

\subsection{题目大意}

给定一个递归定义的生成器,其满足的递归关系为:

$$
x_{n + 1} = ((a x_n + c) \bmod m)
$$

这里 $x$ 是生成的伪随机数序列,而 $m, a, c, x_0$ 则是定义这个生成器所给出的整数常量。

另外,两个整数区间 $[l_1, r_1]$ 和 $[l_2, r_2]$ 也被指定,请你计算下面这个式子的值:

$$
\sum_{i = l_1}^{r_1}\sum_{j = l_2}^{r_2}{(x_i \bmod (x_j + 1))}
$$

一共 $T$ 组数据,保证 $1 \le T \le 10^5,1 \le m \le 10^6, 0 \le a, c, x_0 < m, 0 \le l_1 \le r_1  \le 10^5, 0 \le l_2 \le r_2 \le 10^5,\sum m \le 2 \times 10^6$

\subsection{算法讨论}

看到取模不太方便操作,先化简一下式子:

$$
\begin{aligned}
&\sum_{i=l_1}^{r_1}\sum_{j=l_2}^{r_2} \left(x_i \bmod (x_j+1)\right) \\
=&\sum_{i=l_1}^{r_1}\sum_{j=l_2}^{r_2} \left(x_i-\left\lfloor \frac{x_i}{x_j+1} \right\rfloor (x_j+1)\right) \\
=&\left(\sum_{i=l_1}^{r_1}\sum_{j=l_2}^{r_2} x_i\right)-\left(\sum_{i=l_1}^{r_1}\sum_{j=l_2}^{r_2}\left\lfloor \frac{x_i}{x_j+1} \right\rfloor (x_j+1)\right) \\
\end{aligned}
$$

熟悉那预处理出所有可能的 $x_i$,这个最多只有 $O(m)$ 种,而且是一个 $\rho$ 状分布的

前半部分直接计算就好,后半部分可以枚举 $a=\left\lfloor \frac{x_i}{x_j+1} \right\rfloor,b=x_j+1$,然后统计有多少个 $x_i$ 就行了,即 $x_i \in [ab,(a+1)b-1]$,通过调和级数可得知,时间复杂度为 $O(m \log m)$

\subsection{时间复杂度}

$$
O(m \log m)
$$

\subsection{空间复杂度}

$$
O(m)
$$

\section{\href{http://www.51nod.com/Challenge/Problem.html?problemId=1753}{【51nod 1753】相似子串}}

\subsection{题目大意}

两个字符串相似定义为:

\begin{quotation}
1. 两个字符串长度相等

2. 两个字符串对应位置上有且仅有至多一个位置所对应的字符不相同
\end{quotation}

给定一个字符串 $s$,每次询问两个子串在给定的规则下是否相似

给定的规则指每次给出一些等价关系,如 $a=b,b=c$ 等

注意这里的等价关系具有传递性,即若 $a=b,b=c \Rightarrow a=c$

\subsection{算法讨论}

用并查集把相同的字符合并起来,哈希维护每种字符是否出现的哈希值,然后就可以进行普通的哈希判断了

\subsection{时间复杂度}

$$
O((|s| + T) \Sigma \log n)
$$

\subsection{空间复杂度}

$$
O(|s| \sigma)
$$

\section{\href{https://codeforces.com/problemset/problem/167/D}{【codeforces 167D】 Wizards and Roads}}

\subsection{题目大意}

给定 $n(1 \le 10^5)$ 个随机生成的点,并以某种方式连边,之后 $q(q \le 10^5)$ 次询问 $x$ 在 $[l,r]$ 的所有点的最大匹配

\subsection{算法讨论}

结论:每个点会向在它左右下方的 $y$ 值最大的点连边,之后会成为一棵二叉树,每次询问一个子树的最大匹配

然后就直接暴力连边后,类似于线段树一样的区间查询就行了

\subsection{时间复杂度}

$$
O((q + n) \log n)
$$

\subsection{空间复杂度}

$$
O(n)
$$

\section{\href{https://codeforces.com/problemset/problem/855/C}{【codeforces 855C】Helga Hufflepuff's Cup}}

\subsection{题目大意}

给出一棵 $n(1 \le n \le 10^5)$ 个节点的树,有 $m(1 \le m \le 10^9)$ 种颜色,第 $k(1 \le k \le m)$ 种颜色是特殊颜色,树上最多有 $x(1 \le x \le 10)$ 个特殊颜色点

你需要把整个树染色,且保证特殊颜色节点以下条件: 

\begin{quotation}

1. 与其相连的不能有特殊颜色节点

2. 与其相连的节点的颜色序号必须小于 $k$
	
\end{quotation}

问有多少种满足要求的树,答案模 $10^9+7$

\subsection{算法讨论}

设 $f_{i, j, 0/1/2}$ 表示以 $i$ 为根的子树中,一共有 $j$ 个特殊颜色的节点,其中 $i$ 号节点的颜色小于或等于或大于 $k$,然后暴力转移即可

\subsection{时间复杂度}

$$
O(nx^2)
$$

\subsection{空间复杂度}

$$
O(nx)
$$

\section{\href{https://codeforces.com/problemset/problem/1045/A}{【codeforces 1045A】Last chance}}

\subsection{题目大意}

有 $n$ 个敌方飞船,己方有 $m$ 个武器,有以下三种类型:

\begin{quotation}

1. 能攻击编号属于一个大小为 $k_i$ 的集合的所有飞船

2. 能攻击编号在 $[l_i,r_i]$ 区间内的所有飞船

3. 能攻击三个飞船,编号分别为 $a_i, b_i, c_i$

\end{quotation}

其中,前两种类型的武器每个只能攻击 $1$ 个飞船,第三种每个只能使用 $0$ 次或 $2$ 次

特别性质:每个飞船最多会被一个第三种武器纳入攻击范围

在每个飞船只能被攻击一次的情况下,问最多能攻击到多少个飞船,并输出方案

其中 $n,m \le 5000, \sum k_i \le 10^5$

\subsection{算法讨论}

对于第一种,直接暴力连边,对于第二种,线段树优化建图

对于第三种,首先把所有三元组缩成一个点,然后每个第三种会对答案产生 $+2$ 的贡献

\subsection{时间复杂度}

$$
O((\sum k_i + n \log m) \sqrt {n+m \log m})
$$

\subsection{空间复杂度}

$$
O(\sum k_i + (n + m) \log m)
$$

\section{\href{https://atcoder.jp/contests/agc006/tasks/agc006_c}{【Atcoder Grand Contest 006C】Rabbit Exercise}}

\subsection{题目大意}

数轴上有 $n$ 个点,从 $1$ 到 $n$ 编号,有 $m$ 个操作,每次操作给出一个编号 $a_i(1<a_i<n)$,即把点 $a_i$ 等概率移动到它关于点 $a_i−1$ 的对称点或关于点 $a_i+1$ 的对称点

记顺序执行这 $m$ 个操作为完成 $1$ 次,问完成 $k$ 次后,所有点的坐标的期望值是多少

其中 $3 \le n \le 10^5,x_i \in \mathbb{Z},|x_i| \le 10^9,1 \le m \le 10^5,2 \le a_j \le n-1,1 \le k \le 10^{18}$

\subsection{算法讨论}

首先点 $i$ 被操作后的期望位置是:

$$
E(x_i)=\frac{1}{2}\left(2x_{i-1}-x_i\right)+\frac{1}{2}\left(2x_{i+1}-x_i\right)=x_{i-1}+x_{i+1}-x_{i}
$$

如果原先有五个位置 $x_1, x_2, x_3, x_4, x_5$,那么如果操作了一次 $x_3$,则会变为 $x_1, x_2, x_2+x_4-x_3, x_4, x_5$

考虑差分一下,那么就相当于从 $x_1, x_2-x_1, x_3-x_2, x_4-x_3, x_5-x_4$ 变成了 $x_1, x_2-x_1, x_4-x_3, x_3-x_2, x_5$,也就是交换了相邻两个位置

那么将 $x_i$ 差分,设 $y_i=x_i-x_{i-1}$,那么如果操作 $x_i$,实际上就是交换了 $y_i, y_{i+1}$

那么一次完整的操作就相当于按照顺序进行了一些交换,也就是乘以了一个置换

那么现在只需要求这个置换的 $k$ 次就行了,找出所有的环后每个环走上 $k$ 次就行了

\subsection{时间复杂度}

$$
O(n+m)
$$

\subsection{空间复杂度}

$$
O(n+m)
$$

\section{\href{https://agc002.contest.atcoder.jp/tasks/agc002_d}{【Atcoder Grand Contest 002D】Stamp Rally}}

\subsection{题目大意}

给出一个 $n(1 \le n \le 10^5)$ 个点 $m(1 \le m \le 10^5)$ 条边的无向图,$q(1 \le q \le 10^5)$ 次询问,每次询问给出两个点 $(x, y)$,求包含 $x, y$ 的总大小为 $z$ 的联通块(可能 $x, y$ 不在一个连通块中),使得连通块中的边的序号最大值尽可能小

\subsection{算法讨论}

考虑整体二分,由于修改在询问之前,所以每一层实际上是修改和询问的交替块,于是可以按照 $bfs$ 的顺序进行整体二分

\subsection{时间复杂度}

$$
O((n+q) \log n)
$$

\subsection{空间复杂度}

$$
O(n+q)
$$

\section{\href{https://www.lydsy.com/JudgeOnline/problem.php?id=4870}{【SHOI 2017】组合数问题}}

\subsection{题目大意}

给定四个整数 $n, p, k, r$,求 $\sum_{i=0}^{\infty}{nk \choose ik+r} \bmod p$

其中 $1 \le n \le 10^9, 0 \le r < k \le 50, 2 \le p \le 2^{30}-1$

\subsection{算法讨论}

考虑组合意义,相当于从 $nk$ 个物品中,挑出模 $k$ 等于 $r$ 个物品的方案数

则可以设 $f_{i,j}$ 表示前 $i$ 个物品,挑出模 $k$ 等于 $j$ 个物品的方案数,然后矩阵乘法就行了

\subsection{时间复杂度}

$$
O(k^3 \log (nk))
$$

\subsection{空间复杂度}

$$
O(k^2)
$$

\section{\href{http://uoj.ac/problem/275}{【清华集训 2016】组合数问题}}

\subsection{题目大意}

求如果给定 $n,m$ 和 $k$,对于所有的 $0 \le i \le n, 0 \le j \le \min(i, m)$,有多少对 $(i, j)$ 满足 ${i \choose j}$ 是 $k$ 的倍数

答案对 $10^9+7$ 取模,其中 $1 \le n,m \le 10^{18}, 1 \le k \le 100$,且 $k$ 是个质数

\subsection{算法讨论}

考虑卢卡斯定理:

$$
{n \choose m} \equiv {n \bmod p \choose m \bmod p} {\lfloor\frac{n}{p} \rfloor \choose \lfloor \frac{m}{p} \rfloor} \pmod {p}
$$

即如果把 $n, m$ 都写成 $p$ 进制的数,也就是:

$$
\begin{cases}
&n = \sum_{i=0}^{t} n_i \times p^i \\
&m = \sum_{i=0}^{t} m_i \times p^i \\
\end{cases}
$$

则有:

$$
{n \choose m} \equiv \prod_{i=0}^{t} {n_i \choose m_i} \pmod {p}
$$

那么如果 ${n \choose m} \equiv 0 \pmod {k}$,则相当于 $k$ 进制下 $n$ 的某一位比 $m$ 的对应位小

回到这道题,相当于统计有多少个 $i, j$,满足 $0 \le j \le i \le n$ 且 $j \le m$ 的 $i, j$ 个数,满足 $j$ 的对应位至少有一位比 $i$ 大

一个比较暴力的 $dp$ 是,$f_{L,0/1,0/1,0/1,0/1}$ 表示考虑完长度为 $L$ 的前缀,$i$ 是否顶上 $n$,$j$ 是否顶上 $m$,之前是否存在一位使得 $j$ 大于 $i$

然而这个转移太复杂了,不好写,不妨考虑容斥,即只计算所有 $j \le i$,且 $j$ 的每一位都不超过 $i$ 的合法对数,然后总的对数减去这些就行了

首先把 $m$ 变为 $\min(n,m)$,然后总的对数是:

$$
\begin{aligned}
  &\sum_{i=0}^{n}\sum_{j=0}^{\min(i,m)} 1 \\
= &\left(\sum_{i=0}^{m}\sum_{j=0}^{i} 1\right)+\left(\sum_{i=m+1}^{n}\sum_{j=0}^{m}1\right) \\
= &\left(\sum_{i=0}^{m}i+1\right)+(m+1)(n-m) \\
= &\frac{(1+m)m}{2}+m+1+(m+1)(n-m) \\
= &\frac{(1+m)m}{2}+(m+1)(n-m+1) \\
\end{aligned}
$$

之后设 $f_{L,0/1,0/1}$,表示考虑完长度为 $L$ 的前缀,$i$ 是否顶上 $n$,$j$ 是否顶上 $m$,且 $j$ 的每一位都不能大于 $i$,满足这些条件的合法的 $(i,j)$ 的对数

然后减一下就行了,注意要先取模,因为 $n,m \le 10^{18}$

\subsection{时间复杂度}

$$
O(k^2 \log n)
$$

\subsection{空间复杂度}

$$
O(\log n)
$$

\section{\href{http://uoj.ac/problem/300}{【CTSC 2017】吉夫特}}

\subsection{题目大意}

给定一个长度为 $n$ 的数列 $a_i$ ,求有多少个长度不小于 $2$ 的子序列 $a_{b_1},a_{b_2}, \dots ,a_{b_k}$,满足:

$$
\prod_{i=2}^{k} {a_{b_{i-1}} \choose a_{b_{i}}} \equiv 1 \pmod{2}
$$

其中 $1 \le n \le 211985, a_i \le 233333$,保证 $a_i$ 互不相同

\subsection{算法讨论}

根据卢卡斯定理,可以得知这个序列应该满足二进制下 $a_{b_i} \subseteq a_{b_{i-1}}$

由于 $a_i$ 互不相同,直接暴力的时间复杂度为 $O(3^{18})$,由于 $uoj$ 跑的很快所以就跑过去了

为了简便起见,把 $a$ 先 $reverse$ 一下,之后变为 $a_{b_{i-1}} \subseteq a_{b_i}$ 的问题

考虑均衡一下暴力的复杂度,设 $f_{s,t}$ 表示前 $9$ 位为 $s$,后 $9$ 位为 $t$ 的子集的数字结尾的方案数

那么转移的时候可以先枚举前 $9$ 位的子集来统计答案,更新的时候就可以枚举后 $9$ 位的超集来更新状态

\subsection{时间复杂度}

$$
O(n \times 2^9)
$$

\subsection{空间复杂度}

$$
O(2^{18})
$$

\section{\href{http://192.168.14.17/problem/629}{【EXNR 1C】两开花}}

\subsection{题目大意}

给定一棵大小为 $n$ 的有根树,设 $f_k$ 表示随机选择 $k$ 个不同的点所构成的虚树的大小,求 $f_k$ 的期望值

要求对于所有 $1 \le k \le m$ 的 $f_k$ 都计算,答案模 $998244353$,保证 $1 \le m \le n \le 4 \times 10^5$

\subsection{算法讨论}

大体思路:由于是线性计数,所以只需要考虑每个点会被算多少次就行了

那么一个点如果对答案有贡献,要么它是一个关键点,要么它的两个不同子树中各有一个关键点

设 $a_i$ 表示 $i$ 的子树大小,如果随机选 $k$ 个点的话,那么贡献如下:

\textbf{选了}

如果一个点 $u$ 选了,那么一定会产生一次贡献,至于被算了多少次就乘以组合数就好了

$$
\sum_{i=1}^{n}{n-1 \choose k-1}
$$

\textbf{没选}

也就是至少有两个不同子树中都选了一个,剩下的无所谓,那么就可以大力容斥:子树中除了自己外至少选了一个的方案,减去不合法的,不合法就是只在一个子树中选择

$$
\sum_{i=1}^{n} \left(\sum_{x=1}^{k} {a_i-1 \choose x}{n-a_i \choose k-x}-\sum_{i \to j} \sum_{x=1}^{k}{a_j \choose x}{n-a_i \choose k-x} \right)
$$

那么要计算 $f_k$ 了:

$$
\begin{aligned}
f_k
=&\sum_{i=1}^{n}{n-1 \choose k-1}+\sum_{i=1}^{n} \left(\sum_{x=1}^{k} {a_i-1 \choose x}{n-a_i \choose k-x}-\sum_{i \to j} \sum_{x=1}^{k}{a_j \choose x}{n-a_i \choose k-x} \right) \\
=& n{n-1 \choose k-1}
   +\sum_{i=1}^{n}\sum_{x=1}^{k} {a_i-1 \choose x}{n-a_i \choose k-x}
   -\sum_{j=2}^{n}\sum_{x=1}^{k}{a_j \choose x}{n-a_{fa_j} \choose k-x}
\end{aligned}
$$

设:

$$
\begin{cases}
b_i=a_i-1 \\
c_i=n-a_i \\
d_i=n-a_{fa_i} \\
e_i=c_i+d_i
\end{cases}
$$

则原式为:

$$
f_k=n {n-1 \choose k-1}+\sum_{i=1}^{n}\sum_{x=1}^{k} {b_i \choose x} {c_i \choose k-x}-\sum_{i=2}^{n}\sum_{x=1}^{k}{a_i \choose x}{d_i \choose k-x}
$$

\textbf{左面}

直接暴力就行了

\textbf{中间}

$$
\begin{aligned}
g_k
&=\sum_{i=1}^{n}\sum_{x=1}^{k}{b_i \choose x}{c_i \choose k-x} \\
&=\sum_{i=1}^{n}{b_i+c_i \choose k}-{b_i \choose 0} {c_i \choose k} \\
&=\sum_{i=1}^{n}{n-1 \choose k}-{c_i \choose k} \\
&=n{n-1 \choose k}-\left(\sum_{i=1}^{n} {c_i \choose k}\right) \\
&=n{n-1 \choose k}-\left(\sum_{i=1}^{n} {n-a_i \choose k}\right) \\
\end{aligned}
$$

\textbf{右面}

$$
\begin{aligned}
h_k
= &\sum_{i=2}^{n}\sum_{x=1}^{k}{a_i \choose x}{d_i \choose k-x} \\
= &\sum_{i=2}^{n}{n+a_i-a_{fa_i} \choose k}-{n-a_{fa_i} \choose k} \\
\end{aligned}
$$

至此可以发现只需要计算下面这个东西就行了:

$$
\begin{aligned}
p_k
=&\sum_{i=1}^{n}{a_i \choose k} \\
\end{aligned}
$$

设 $c_k=\sum_{i=1}^{n}[a_i=k]$

$$
\begin{aligned}
p_k
= &\sum_{i \in V} {v \choose k}c_v \\
= &\sum_{i \in V} \frac{v!}{k!(v-k)!}c_v
\end{aligned}
$$

注意到可行的 $a_i$ 范围是 $0 \le a_i \le 2n$,也就是说:

$$
\begin{aligned}
k!p_k
=&\sum_{i=0}^{2n}i!c_i\frac{1}{(i-k)!} \\
=&\sum_{i=0}^{2n}i!c_i\frac{1}{(i-k)!}
\end{aligned}
$$

设 $X_i=i!c_i,Y_i=(2n-i)!,Z_k=k!p_k$,则:

$$
Z_k=\sum_{i=0}^{2n}X_iY_{2n-i+k}
$$

令 $\forall t > 2n,X_{t}=0$,则有:

$$
\begin{aligned}
Z_{2n+k}
=&\sum_{i=0}^{2n+k}X_iY_{2n-i+k} \\
=&\sum_{i+j=2n+k}X_iY_{j} \\
\end{aligned}
$$

于是就是一个普通的卷积形式了

\subsection{时间复杂度}

$$
O(n \log n)
$$

\subsection{空间复杂度}

$$
O(n \log n)
$$

\section{\href{https://codeforces.com/problemset/problem/1110/C}{【codeforces 1110C】Meaningless Operations}}

\subsection{题目大意}

定义函数 $f(a)$ 的值为:

$$
f(a)=\max_{0<b<a}\gcd(a\oplus b,a\ \&\ b)
$$

给出 $q$ 个询问,每个询问为一个整数 $a_i$,​你需要对于每个询问,求出 $f(a_i)$ 的值

数据范围:$1\le q \le 1000, 2\le a_i\le 2^{25}-1$

\subsection{算法讨论}

一道十分巧妙的构造题

首先 $\gcd(a, b) \le \min(a, b),a \& (a-t) \le a$

考虑一个 $a$,那么可以把它的最高位及以下都取翻来得到 $b$,比如说如果 $a=(11010)_2$,那么 $b=(00101)_2$,这样 $\gcd(a \oplus b, a \& b)=\gcd(a \oplus b, 0)=\gcd(a \oplus b)$,且这个 $a \oplus b$ 是能达到的最大值了

但是有一些例外,比如说 $b=0$,此时 $a=2^k-1$,那么可以直接打表得到这些 $a$ 对应的值就行了

\subsection{时间复杂度}

$$
O(\log a)
$$

\subsection{空间复杂度}

$$
O(\log a)
$$

\section{\href{https://vjudge.net/problem/TopCoder-12118}{【Topcoder SRM550 DIV1 Hard】ConversionMachine}}

\subsection{题目大意}

给定两个字符串 $S,T$,保证只由 $c,a,t$ 三个字母组成,且 $|S|=|T|$,同时有三个变量 $c',a',t'$,每次可以选择一个 $S$ 中的字符 $p$,然后变为 $q$,有如下代价:

\begin{quotation}

1. $p=c, q=a$,代价是 $c'$

2. $p=a, q=t$,代价是 $a'$

3. $p=t, q=c$,代价是 $t'$

\end{quotation}

求有多少种操作方式,使得 $S$ 能变为 $T$,且代价之和不超过 $n$,两种方式不同当且仅当操作序列的长度不同,或者某一次操作的位置不同

其中 $|S|=|T| \le 13, 1 \le c', a', t' \le 10^9, 0 \le n \le 10^9$

\subsection{算法讨论}

设 $L=|S|=|T|$,如果直接矩阵乘法的话,状态数是 $O(3^{L})$ 的,考虑压缩一些状态,发现每一个状态可以合并为 $(x, y, z)$,分别表示模 $3$ 余 $0, 1, 2$ 的位置的个数

然后状态数就压缩到了 $O(L^3)$,于是可以直接矩阵快速幂来做了

\subsection{时间复杂度}

$$
O(L^3 \log n)
$$

\subsection{空间复杂度}

$$
O(L^2)
$$

\section{\href{https://lydsy.com/JudgeOnline/problem.php?id=4669}{【bzoj 4669】抢夺}}

\subsection{题目大意}

大战将至,$C$ 国决定实行计划经济

$C$ 国西部总共有 $n$ 个城市,编号为 $0 \sim n-1$,以及 $m$ 条道路,道路是单向的

其中城市 $0$ 是一个大城市,里面住着 $k$ 个人,而城市 $n-1$ 是一个农业城市

现在所有城市 $0$ 的居民都需要到城市 $n-1$ 去领取食物

由于担心体力不支,所以居民都会采取开车的形式出行

但道路不是无限宽的,对于一条道路,会有 $c_i$ 的限制,表示在同一天内,最多只能有 $c_i$ 辆车同时在这条道路上行驶

一条道路的长度为 $1$,每辆车的行驶速度也可以假定为 $1$ 每天

城市 $n-1$ 会在每个居民都到达后马上开始发放食物

现在板板想知道,假如在最优安排下,居民最早能在多少天后领到食物

假如没有居民那就不需要发放食物,默认为第 $0$ 天

\subsection{算法讨论}

二分答案后费用流贪心判定

那么为什么是对的……首先第一次增广之后,第二次增广的时候只能沿着第一次没有填满的那些地方走,因为第一次的增广是连续的,所以这么做是对的……

\subsection{时间复杂度}

$$
O(nmf \log n)
$$

\subsection{空间复杂度}

$$
O(n+m)
$$

\section{\href{https://agc005.contest.atcoder.jp/tasks/agc005_f}{【Atcoder Grand Contest 005F】Many Easy Problems}}

\subsection{题目大意}

给定一棵大小为 $n$ 的无根树,定义 $f(i)$,对于所有大小为 $i$ 的点集,求出能够包含它的最小连通块大小之和

对于 $1 \le i \le n$ 的所有 $i$,求出 $f(i)$

其中 $1 \le n \le 2 \times 10^5$,答案模 $924844033$

\subsection{算法讨论}

考虑贡献,假设一共选择 $k$ 个

\textbf{选了}

$$
\sum_{i=1}^{n} {n-1 \choose k-1}
$$

\textbf{没选}

$$
\sum_{i=1}^{n}\sum_{x=1}^{k}{a_i-1 \choose x}{n-a_i \choose k-x}
$$

\textbf{算总账}

$$
\begin{aligned}
f_k
=&\sum_{i=1}^{n} {n-1 \choose k-1}+\sum_{i=1}^{n}\sum_{x=1}^{k}{a_i-1 \choose x}{n-a_i \choose k-x}-\sum_{i \to j} {a_j \choose k} \\
=&n{n-1 \choose k-1}+\sum_{i=1}^{n} {n-1 \choose k}-{n-a_i \choose k} -\sum_{i=2}^{n} {a_i \choose k}  \\
=&n \left({n-1 \choose k-1}+{n-1 \choose k}\right)-\sum_{i=1}^{n} {n-a_i \choose k}+ {a_i \choose k}[i \ge 2] \\
=&n {n \choose k}-\sum_{i=1}^{n} {n-a_i \choose k} + {a_i \choose k} [i \ge 2] \\
\end{aligned}
$$

\subsection{时间复杂度}

$$
O(n \log n)
$$

\subsection{空间复杂度}

$$
O(n)
$$

\section{\href{https://www.lydsy.com/JudgeOnline/problem.php?id=2006}{【NOI 2010】超级钢琴}}

考虑把所有子串都表示出来,枚举右端点 $r$,那么左端点取值范围是 $[r-R+1,r-L+1]=[a, b]$,之后可以得到一个左端点,使得权值和最大,记这个值为 $sum$,那么可以用一个四元组来表示这些子串:$(sum, r, a, b)$

之后每次找一个 $sum$ 最大的子串,并累加到答案上,之后删除这个四元组,并求得任意一个 $l$,使得 $[l, r]$ 取到最大权值且 $l$ 合法,然后新增两个四元组:$(sum_a, r, a, l-1)$ 和 $(sum_b, r, l + 1, b)$

\subsection{时间复杂度}

$$
O((n+k) \log n)
$$

\subsection{空间复杂度}

$$
O(n+k)
$$

%\section{\href{}{}}
%\subsection{题目大意}
%\subsection{算法讨论}
%\subsection{时间复杂度}
%\subsection{空间复杂度}

\end{document}
