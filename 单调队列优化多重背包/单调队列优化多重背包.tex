%!TEX program = xelatex

\documentclass[UTF8]{article}
\author {sjzez czy}
\title {单调队列优化多重背包}
\date{2019.2.20}
\usepackage[UTF8]{ctex}

\usepackage{listings}
\usepackage{fontspec}
\usepackage{amsmath}
\usepackage{amssymb}
\usepackage{geometry}
\usepackage{setspace}
\usepackage{abstract}
\usepackage{graphicx}
\usepackage{verbatim}
\usepackage[colorlinks,linkcolor=black,citecolor=black]{hyperref}
\renewcommand{\baselinestretch}{1.5}
\geometry{a4paper,left=2.7cm,right=2.7cm,top=2.7cm,bottom=2.7cm}
\setmonofont{Consolas}

\usepackage{xcolor}
% 定义可能使用到的颜色
\definecolor{CPPLight}  {HTML} {686868}
\definecolor{CPPSteel}  {HTML} {888888}
\definecolor{CPPDark}   {HTML} {262626}
\definecolor{CPPBlue}   {HTML} {4172A3}
\definecolor{CPPGreen}  {HTML} {487818}
\definecolor{CPPBrown}  {HTML} {A07040}
\definecolor{CPPRed}    {HTML} {AD4D3A}
\definecolor{CPPViolet} {HTML} {7040A0}
\definecolor{CPPGray}  {HTML} {B8B8B8}
\lstset{
    columns=fixed,
    numbers=left,                                        % 在左侧显示行号
    frame=none,                                          % 不显示背景边框
    backgroundcolor=\color[RGB]{245,245,244},            % 设定背景颜色
    keywordstyle=\color[RGB]{40,40,255},                 % 设定关键字颜色
    numberstyle=\footnotesize\color{darkgray},           % 设定行号格式
    commentstyle=\it\color[RGB]{0,96,96},                % 设置代码注释的格式
    stringstyle=\rmfamily\slshape\color[RGB]{128,0,0},   % 设置字符串格式
    showstringspaces=false,                              % 不显示字符串中的空格
    language=c++,                                        % 设置语言
    morekeywords={alignas,continute,friend,register,true,alignof,decltype,goto,
    reinterpret_cast,try,asm,defult,if,return,typedef,auto,delete,inline,short,
    typeid,bool,do,int,signed,typename,break,double,long,sizeof,union,case,
    dynamic_cast,mutable,static,unsigned,catch,else,namespace,static_assert,using,
    char,enum,new,static_cast,virtual,char16_t,char32_t,explict,noexcept,struct,
    void,export,nullptr,switch,volatile,class,extern,operator,template,wchar_t,
    const,false,private,this,while,constexpr,float,protected,thread_local,
    const_cast,for,public,throw,std},
}

\begin{document}

\maketitle

\tableofcontents

\newpage

\section{模型描述}

有 $n$ 种物品,第 $i$ 种物品有 $c_i$ 个,每个价值为 $v_i$,单价 $w_i$

现在有 $m$ 块钱,需要买价格之和不超过 $m$ 块钱的物品,使得价值之和最大

\section{解法}

设 $f_{i,j}$ 表示考虑了前 $i$ 种物品,还剩 $j$ 块钱的最大价值之和,有转移:

$$
f_{i,j}=\max_{k=0}^{c_i} \left(f_{i-1,j-k \cdot w_i}+ k \cdot v_i \right)
$$

考虑在模 $w_i$ 剩余系下做这个背包,即设 $j=a \cdot w_i+b$,有转移:

$$
f_{i,j}=\max_{k=0}^{c_i} \left( f_{i-1,(a-k)w_i+b} + k \cdot v_i \right)
$$

设 $t=a-k​$,有转移:

$$
\begin{aligned}
f_{i,j}
= & \max_{k=0}^{c_i} \left( f_{i-1,t \cdot w_i+b} + (a-t) \cdot v_i \right) \\
= & \max_{k=0}^{c_i} \left( f_{i-1,t \cdot w_i+b} - t \cdot v_i \right) + a \cdot v_i\\
\end{aligned}
$$

同时有约束:

$$
\begin{cases}
a-t \ge 0 \\
k \le c_i \Rightarrow k=a-t \le c_i
\end{cases}
$$

即:

$$
a-c_i \le t \le a
$$

于是可以先枚举 $b$,然后枚举 $a$,最后单调队列枚举 $t$

时间复杂度为 $O(nm)$

\section{代码实现}

\begin{lstlisting}[language=C++,numberstyle=\tiny,basicstyle=\ttfamily]
#include <bits/stdc++.h>
using namespace std;
typedef long long ll;
const int N = 110, M = 4e4 + 10, Q = 1e6 + 10;
int f[N][M], n, m;
int que[Q], l, r, que_val[Q];
void add(int pos, int value, int weight, int count) {
    for(int b = 0 ; b <= min(m, weight - 1) ; ++ b) {
        l = 1, r = 0;
        for(int a = 0, t = 0 - count ; a * weight + b <= m ; ++ a) {
            // t in [a - count, a]
            int j = a * weight + b;
            while(l <= r && que[l] < a - count) ++ l;
            while(t <= a) {
                if(t * weight + b >= 0) {
                    int val = f[pos - 1][t * weight + b] - t * value;
                    while(l <= r && que_val[r] <= val) -- r;
                    ++ r;
                    que[r] = t;
                    que_val[r] = val;
                }
                ++ t;
            }
            if(l <= r) {
                f[pos][j] = max(f[pos][j], que_val[l] + a * value);
            }
        }
    }
    for(int j = 0 ; j < m ; ++ j) {
        assert(f[pos][j] <= f[pos][j + 1]);
    }
}
int main() {
    cin >> n >> m;
    for(int i = 1 ; i <= n ; ++ i) {
        int v, w, c;
        cin >> v >> w >> c;
        add(i, v, w, c);
    }
    cout << f[n][m] << endl;
}    
\end{lstlisting}

\section{参考资料}

\href{https://blog.csdn.net/lvzelong2014/article/details/79594011}{lvzelong2014《bzoj4182: Shopping 树上多重背包+点分治》}

\end{document}
