%!TEX program = xelatex

\documentclass[UTF8]{article}
\author {sjzez czy}
\title {一类通过生成函数求线性递推式的方法}
\date{2019.1.22}
\usepackage[UTF8]{ctex}
\usepackage{listings}
\usepackage{fontspec}
\usepackage{ctex}
\usepackage{amsmath}
\usepackage{amssymb}
\usepackage{geometry}
\usepackage{setspace}
\usepackage{abstract}
\usepackage{graphicx}

% \setmonofont{Consolas}

\usepackage[colorlinks,linkcolor=black,citecolor=black]{hyperref}
\renewcommand{\baselinestretch}{1.5}
\geometry{a4paper,left=2.7cm,right=2.7cm,top=2.7cm,bottom=2.7cm}


\begin{document}

\maketitle

\begin{abstract}
    在一类低阶卷积递推中,常用方法是多项式开根来优化递推,但在某些特殊情况下,可以实现线性递推来求解,使得时间复杂度变为线性
    \newline
    \centering
    \textbf{关键字:} 卷积、形式幂级数、生成函数、多项式开根
\end{abstract}

\newpage
\tableofcontents

\newpage

\section{定义}

\subsection{卷积}

对于一个矩阵 $\{a_{n,m}\}$,定义其 $n$ 阶卷积 $\{c_k\}$ 为:

$$
c_k=\sum_{\sum_{j=1}^{n}p_j=k}\prod_{i=1}^{n} a_{i,p_i}
$$

\subsection{卷积递推}

定义一个序列 $\{f_k\}$ 是关于 $g_k,h_k,\{a_{u,v}\}$ 的 $u$ 阶卷积递推,意味着:

$$
f_k=g_k+h_k\sum_{\sum_{j=1}^{u}p_j=k}\prod_{i=1}^{u}a_{i,p_i}
$$

\subsection{形式幂级数}

对于一个序列 $\{a_n\}$,定义它的形式幂级数 $A(x)$ 为:

$$
A(x)=\sum_{n=0}^{\infty}a_nx^n
$$

在这里,并不关注 $x$ 的取值,只是用它的一个形式,但如果将其转化为封闭形式后,需要保证 $a_0=A(0)$ 成立

在本文中,不加区分形式幂级数与生成函数的区别,形式幂级数拥有一些常见的算术运算方式

\subsubsection{相等}

定义 $A(x)=B(x)$,意味着:

$$
\forall n \ge 0,a_n=b_n
$$

\subsubsection{加法}

定义 $C(x)=A(x)+B(x)$,意味着:

$$
\sum_{n=0}^{\infty}c_nx^n=\sum_{n=0}^{\infty}(a_n+b_n)x^n
$$

\subsubsection{乘法}

定义 $C(x)=A(x) \times B(x)$,意味着:

$$
\sum_{n=0}^{\infty}c_nx^n=\sum_{n=0}^{\infty}x^n \sum_{i+j=n} a_i \times b_j
$$

\section{引理}

\subsection{形式幂级数的导数}

由于 $x^n$ 的导数为 $nx^{n-1}$,由此可得对于形式幂级数 $A(x)$ 来说,它的导数为:

$$
A ^\prime (x)=\sum_{n=0}^{\infty} a_{n+1}(n+1)x^n
$$

于是可以推出如下结论:

$$
\begin{aligned}
    & F(x)=A(x)^{p} \\
    \Rightarrow & F ^\prime (x)=pA(x)^{p-1}A ^\prime (x) \\
    \Rightarrow & A(x)F ^\prime (x)=pA(x)^pA ^\prime (x) \\
    \Rightarrow & A(x)F ^\prime (x)=pF(x)A ^\prime (x)
\end{aligned}
$$

同时可以得知,如果已知 $F(0)$ 和 $F ^\prime (x)$,可以还原出 $F(x)$

\subsection{形式幂级数的二次卷积}

考虑形如如下的一类递推:

$$
f_k=g_k+h_k\sum_{i+j=k}f_if_j
$$

在进行一些整理后,它们的形式幂级数可以写作:

$$
A(x)F^2(x)+B(x)F(x)+C(x)=0
$$

若其为二次方程,则通过二次方程求根公式,可以得到:

$$
F(x)=\frac{-B(x) \pm \sqrt{B^2(x)-4A(x)C(x)}}{2A(x)}
$$

在确定完正负号后,可以化作:

$$
2A(x)F(x)=-B(x) \pm \sqrt{B^2(x)-4A(x)C(x)}
$$

因此只需要考虑一类形如 $\sqrt{A(x)}$ 的生成函数即可

当 $A(x),B(x),C(x)$ 均为低阶多项式时,可以进行一些化简后,得到 $\{f_n\}$ 的线性递推式

设 $F(x)=\sqrt{A(x)}$,则有:

$$
\begin{cases}
    F ^\prime (x)=\frac{A ^\prime (x)}{2\sqrt{A(x)}}=\frac{A ^\prime (x)}{2} G(x) \\
    G ^\prime (x)=\left(\frac{1}{\sqrt{A(x)}}\right) ^\prime =-\frac{A ^\prime (x)}{2A(x)}G(x)
\end{cases}
$$

即:

$$
\begin{cases}
    A ^\prime (x)G(x)=-2G ^\prime (x)A(x) \\
    2F ^\prime (x)=A ^\prime (x)G(x)
\end{cases}
$$

展开后可以得到:

$$
\begin{cases}
    A ^\prime (x)\sum_{n=0}^{\infty}g_nx^n=-2A(x)\sum_{n=0}^{\infty}g_{n+1}(n+1)x^n \\
    \sum_{n=0}^{\infty}2f_{n+1}(n+1)x^n=A ^\prime (x)\sum_{n=0}^{\infty}g_nx^n
\end{cases}
$$

在依次进行线性递推后,可以得到 $\{f_n\}$

\section{实例}

\subsection{Large Schröder numbers}

已知 \textbf{Large Schröder numbers} 的形式幂级数为 $F(x)=\frac{1-x-\sqrt{1-6x^+x^2}}{2x}$,求其线性递推式

设 $G(x)=\sqrt{1-6x+x^2}$

\subsubsection{求 $G(x)$}

因为:

$$
\begin{aligned}
&(1-6x+x^2)G'(x)=\frac{1}{2}G(x)(1-6x+x^2)' \\
\Rightarrow &(1-6x+x^2)G'(x)=G(x)(x-3) \\
\Rightarrow &\sum_{n=0}^{\infty}g_{n+1}(n+1)x^{n}-\sum_{n=1}^{\infty}6g_{n}nx^n+\sum_{n=2}^{\infty}g_{n-1}(n-1)x^n=\sum_{n=1}^{\infty}g_{n-1}x^n-\sum_{n=0}^{\infty} 3g_nx^n
\end{aligned}
$$

于是有:

$$
\begin{cases}
g_0=G(0)=1 \\
g_{0+1}(0+1)=-3g_0 \\
g_{1+1}(1+1)-6g_11=g_{1-1}-3g_1 \\
g_{n+1}(n+1)-6g_nn+g_{n-1}(n-1)=g_{n-1}-3g_n \quad (n \ge 2)
\end{cases}
$$

那么:

\textbf{初值}

$$
\begin{cases}
g_0=1 \\
g_1=-3g_0=-3 \\
2g_2-6g_1=g_0-3g_1 \Rightarrow g_2=-4
\end{cases}
$$

\textbf{其它情况}

$$
\begin{aligned}
&g_{n+1}(n+1)+(3-6n)g_n+g_{n-1}(n-2)=0 \quad & (n \ge 2) \\
\Rightarrow &g_{n+1}=\frac{(6n-3)g_n+g_{n-1}(2-n)}{n+1} \quad & (n \ge 2) \\
\Rightarrow &g_{n}=\frac{(6n-9)g_{n-1}+(3-n)g_{n-2}}{n} \quad & (n \ge 1)
\end{aligned}
$$

\subsubsection{求 $F(x)$}

因为 $2xF(x)=1-x-G(x)$,则有:

$$
\sum_{n=1}^{\infty}2f_{n-1}x^n=1-x-g_0-g_1x-\sum_{n=2}^{\infty}g_nx^n
$$

那么:

\textbf{初值}

$$
2f_{1-1}=-1-g_1 \Rightarrow f_0=1
$$

\textbf{其它情况}

$$
\begin{aligned}
&2f_{n-1}=-g_n \quad &(n \ge 2) \\
\Rightarrow &f_{n}=-\frac{g_{n+1}}{2} \quad & (n \ge 1)
\end{aligned}
$$

\subsection{Motzkin numbers}

已知 \textbf{Motzkin numbers} 的形式幂级数为 $F(x)=\frac{1-x-\sqrt{1-2x-3x^2}}{2x^2}$,求其线性递推式

\subsection{Catalan numbers}

已知 \textbf{Catalan numbers} 的形式幂级数为 $F(x)=\frac{1-\sqrt{1-4x}}{2x}$,求其线性递推式

\begin{thebibliography}{}
    \bibitem[1]{REF1}\href{https://zhuanlan.zhihu.com/p/31457805}{生成函数及应用}, ACdreamer
    \bibitem[2]{REF2}\href{https://zhuanlan.zhihu.com/p/54318231}{迷思:寻找母函数运算的组合意义(一)}, EntropyIncreaser
\end{thebibliography}

\end{document}
